\documentclass[final,12pt, notitlepage]{article}

%% Packages
% Math
\usepackage{amsmath}
\usepackage{amsthm}
\usepackage{amssymb}
\usepackage{bbm}
%\usepackage{empheq}
%\usepackage{MnSymbol} % for \dashedrightarrow - incompatible with amssymb?

% Automatic line breaking in equations
%\usepackage{flexisym}
%\usepackage{breqn}
%\usepackage{mathtools}

% Graphics & Figures & Tables
\usepackage{graphicx}
%\usepackage[all]{xypic} - this is incompatible with diagrams!
\usepackage[margin=40pt,font=small,bf]{caption} 
%\usepackage{diagrams}
\usepackage{multirow}

% Misc
%\usepackage[shortlabels]{enumitem}
%\usepackage{enumerate}
\renewcommand\labelenumi{(\roman{enumi})}
\renewcommand\theenumi\labelenumi

% Tensors
\usepackage{tensind}
\tensordelimiter{?}
\tensorformat{}

%% Math Operators
\DeclareMathOperator{\PV}{p.v.}
\DeclareMathOperator{\Hom}{Hom}
\DeclareMathOperator{\End}{End}
\DeclareMathOperator{\Id}{Id}
\DeclareMathOperator{\diag}{diag}
\DeclareMathOperator{\cchar}{char}
\DeclareMathOperator{\tr}{tr}
\DeclareMathOperator{\Tr}{Tr}
\DeclareMathOperator{\Sp}{sp}
\DeclareMathOperator{\sgn}{sgn}
\DeclareMathOperator{\ev}{ev}
\DeclareMathOperator{\lcm}{lcm}
\DeclareMathOperator{\spec}{spec}
\DeclareMathOperator{\Ext}{Ext}
\DeclareMathOperator{\Tor}{Tor}
\DeclareMathOperator{\Ad}{Ad}
\DeclareMathOperator{\Vol}{Vol}
\DeclareMathOperator{\vol}{vol}
\DeclareMathOperator{\supp}{supp}
\DeclareMathOperator{\support}{support}
\DeclareMathOperator{\spt}{spt}
\DeclareMathOperator{\irr}{irr}
\DeclareMathOperator{\Span}{span}
\DeclareMathOperator{\im}{im}
\DeclareMathOperator{\IM}{Im}
\DeclareMathOperator{\Ker}{Ker}
\DeclareMathOperator{\Aut}{Aut}
\DeclareMathOperator{\dom}{dom}
\DeclareMathOperator{\Char}{char}
\DeclareMathOperator{\rank}{rank}
\DeclareMathOperator{\Sing}{Sing}
\DeclareMathOperator{\Gr}{Gr}
\DeclareMathOperator{\HP}{HP}
\DeclareMathOperator{\HF}{HF}
\DeclareMathOperator{\codim}{codim}
\DeclareMathOperator{\arccot}{arccot}
\DeclareMathOperator{\sign}{sign}
\DeclareMathOperator{\grad}{grad}
\DeclareMathOperator{\Int}{Int}
\DeclareMathOperator{\conv}{conv}
\DeclareMathOperator{\ord}{ord}
\DeclareMathOperator{\dist}{dist}
\DeclareMathOperator{\Frac}{frac}
\DeclareMathOperator{\rad}{rad}
\DeclareMathOperator{\graph}{graph}
\DeclareMathOperator{\ind}{ind}
\DeclareMathOperator{\Cl}{Cl}
\DeclareMathOperator{\coker}{coker}
\DeclareMathOperator{\re}{Re}
\DeclareMathOperator{\EV}{EV}
%\DeclareMathOperator{\Div}{div}
\DeclareMathOperator{\Pic}{Pic}
\DeclareMathOperator{\Res}{Res}
\DeclareMathOperator{\area}{area}
\DeclareMathOperator{\depth}{depth}
\DeclareMathOperator{\Var}{Var}
\DeclareMathOperator{\Cov}{Cov}
\DeclareMathOperator{\Lip}{Lip}
\DeclareMathOperator{\Out}{Out}
\DeclareMathOperator{\Inn}{Inn}
\DeclareMathOperator{\ad}{ad}
\DeclareMathOperator{\Unif}{Unif}
\DeclareMathOperator{\Poisson}{Poisson}
\DeclareMathOperator{\Binomial}{Binomial}
\DeclareMathOperator{\Exponential}{Exponential}
\DeclareMathOperator{\Normal}{Normal}
\DeclareMathOperator{\curl}{curl}
\DeclareMathOperator{\Area}{Area}
\DeclareMathOperator{\cof}{cof}
\DeclareMathOperator{\diam}{diam}
\DeclareMathOperator{\Mat}{Mat}
\DeclareMathOperator{\osc}{osc}
\DeclareMathOperator{\Ent}{Ent}

\DeclareMathOperator*{\argmax}{arg\,max}
\DeclareMathOperator*{\argmin}{arg\,min}

% hyperbolic
\DeclareMathOperator{\sech}{sech}
\DeclareMathOperator{\csch}{csch}
\DeclareMathOperator{\arsec}{arsec}
\DeclareMathOperator{\arcot}{arcot}
\DeclareMathOperator{\arcsc}{arcsc}
\DeclareMathOperator{\arcosh}{arcosh}
\DeclareMathOperator{\arsinh}{arsinh}
\DeclareMathOperator{\artanh}{artanh}
\DeclareMathOperator{\arsech}{arsech}
\DeclareMathOperator{\arcsch}{arcsch}
\DeclareMathOperator{\arcoth}{arcoth} 

%% Some useful commands

% greek letters
\newcommand{\al}{\alpha}
\newcommand{\e}{\epsilon}
\newcommand{\del}{\delta}
\newcommand{\lam}{\lambda}
\newcommand{\om}{\omega}
\newcommand{\Lam}{\Lambda}
\newcommand{\Om}{\Omega}

% mathbb letters
\newcommand{\LL}{\mathbb{L}}
\newcommand{\R}{\mathbb{R}}
\newcommand{\N}{\mathbb{N}}
\newcommand{\M}{\mathbb{M}}
\newcommand{\Q}{\mathbb{Q}}
\newcommand{\Z}{\mathbb{Z}}
\newcommand{\C}{\mathbb{C}}
\newcommand{\A}{\mathbb{A}}
\newcommand{\PP}{\mathbb{P}}
\newcommand{\GG}{\mathbb{G}}
\newcommand{\KK}{\mathbb{K}}
\newcommand{\K}{\mathbbm{k}}
\newcommand{\U}{\mathbb{U}}
\newcommand{\VV}{\mathbb{V}}
\newcommand{\HH}{\mathbb{H}}
\newcommand{\F}{\mathbb{F}}
\newcommand{\one}{\mathbbm{1}}
\newcommand{\T}{\mathbb{T}}
\newcommand{\UD}{\mathbb{D}}
\newcommand{\E}{\mathbb{E}}
\newcommand{\SB}{\mathbb{S}}
\newcommand{\I}{\mathbb{I}}

% mathcal letters
\newcommand{\mcB}{\mathcal{B}}
\newcommand{\mcL}{\mathcal{L}}
\newcommand{\mcK}{\mathcal{K}}
\newcommand{\mcO}{\mathcal{O}}
\newcommand{\mcF}{\mathcal{F}}
\newcommand{\mcT}{\mathcal{T}}
\newcommand{\mcS}{\mathcal{S}}
\newcommand{\mcG}{\mathcal{G}}
\newcommand{\mcX}{\mathcal{X}}
\newcommand{\mcP}{\mathcal{P}}
\newcommand{\mcY}{\mathcal{Y}}
\newcommand{\mcQ}{\mathcal{Q}}
\newcommand{\mcM}{\mathcal{M}}
\newcommand{\mcN}{\mathcal{N}}
\newcommand{\mcC}{\mathcal{C}}
\newcommand{\mcD}{\mathcal{D}}
\newcommand{\mcA}{\mathcal{A}}
\newcommand{\mcE}{\mathcal{E}}
\newcommand{\mcH}{\mathcal{H}}
\newcommand{\mcR}{\mathcal{R}}
\newcommand{\mcW}{\mathcal{W}}
\newcommand{\mcI}{\mathcal{I}}

% mathfrak letters
\newcommand{\mfp}{\mathfrak{p}}
\newcommand{\mfq}{\mathfrak{q}}
\newcommand{\mfm}{\mathfrak{m}}
\newcommand{\mfn}{\mathfrak{n}}

% tilde letters
\newcommand{\tgamma}{\tilde{\gamma}}
\newcommand{\tX}{\tilde{X}}
\newcommand{\tY}{\tilde{Y}}
\newcommand{\tZ}{\tilde{Z}}
\newcommand{\tN}{\tilde{N}}
\newcommand{\tM}{\tilde{M}}
\newcommand{\tOmega}{\tilde{\Omega}}
\newcommand{\tg}{\tilde{g}}

% groups
\newcommand{\ZN}{{\Z/N\Z}}
\newcommand{\Zn}{{\Z/n\Z}}

% arrows
\newcommand{\into}{\hookrightarrow}
\newcommand{\onto}{\twoheadrightarrow}
\newcommand{\ra}{\rightarrow}
\newcommand{\mra}{\mapsto}
\newcommand{\pra}{\rightharpoonup}      % partial function
%\newcommand{\pra}{\dashedrightarrow}      % partial function - needs MnSymbol
\newcommand{\lra}{\leftrightarrow}
\newcommand{\ira}{\hookrightarrow}        % injection
\newcommand{\sra}{\twoheadrightarrow} % surjection
\newcommand{\xra}{\xrightarrow}
\newcommand{\wka}{\rightharpoonup}     % weak convergence
\newcommand{\wsa}{\stackrel{*}{\rightharpoonup}} % weak-* convergence
\newcommand{\upto}{\uparrow}
\newcommand{\dnto}{\downarrow}

% misc.
\newcommand{\mc}{\mathcal}
\newcommand{\mf}{\mathfrak}
\newcommand{\mbf}{\mathbf}
\newcommand{\mbg}{\boldsymbol} % for greek letters
\newcommand{\dual}{\widehat}
\newcommand{\pmat}[1]{\begin{pmatrix} #1 \end{pmatrix}}
\newcommand{\inp}[1]{\langle #1 \rangle} % inner product (angle brackets) - one argument
\newcommand{\ip}[2]{\langle #1, #2 \rangle} % inner product (angle brackets) - two arguments
\newcommand{\st}{\ \text{s.t.}\ }
\newcommand{\V}[1]{\mathbf{#1}}
\newcommand{\impl}{\Rightarrow}
\newcommand{\qimpl}{\qquad\Rightarrow\qquad}
\newcommand{\imply}{$\Rightarrow$}
\newcommand{\LRI}{$\Rightarrow$} % left to right implication
\newcommand{\RLI}{$\Leftarrow$} % right to left implication
\newcommand{\equv}{$\Leftrightarrow$}
\newcommand{\miff}{\Leftrightarrow}
\newcommand{\isom}{\cong}
\newcommand{\gen}[1]{\langle #1 \rangle}
\newcommand{\ds}{\displaystyle}
\newcommand{\imod}[1]{\,\text{(mod $#1$)}}
\newcommand{\tp}{\otimes} % tensor product
\newcommand{\sm}{\backslash}
\newcommand{\paren}[1]{\left( {#1} \right) }
\newcommand{\pfrac}[2]{\paren{ \frac{#1}{#2} } }
\newcommand{\pfr}{\pfrac}
\newcommand{\brak}[1]{\left[ {#1} \right] }
\newcommand{\brac}[1]{\left\{ {#1} \right\} }
\newcommand{\Abs}[1]{\left| {#1} \right| }
\newcommand{\Norm}[1]{\Abs{\Abs{ #1 } }}
\newcommand{\conj}{\overline}
\newcommand{\bd}{\partial}
\newcommand{\w}{\wedge}
\newcommand{\sub}{\subset}
\newcommand{\LS}[2]{\paren{ \frac{#1}{#2} }} % Legendre symbol
\newcommand{\BLS}[2]{\left[ \frac{#1}{#2} \right]} % Legendre symbol in brackets
\newcommand{\av}[1]{\overrightarrow{#1}} % vector with arrow
\newcommand{\evat}[2]{\left. {#1} \right|_{#2} }
\newcommand{\rbar}[1]{\left. {#1} \right| }
\newcommand{\inv}[1]{{\frac{1}{#1}}}

% partial derivatives (based on http://www.sci.usq.edu.au/staff/robertsa/LaTeX/ltxmaths.html)
\newcommand{\D}[2]{\frac{\partial #2}{\partial #1}} 
\newcommand{\DD}[2]{\frac{\partial^2 #2}{\partial #1^2}}
\newcommand{\DK}[3]{\frac{\partial^{#1} {#3}}{\partial #2^{#1}} }

% ordinary derivatives
\newcommand{\OD}[2]{\frac{d {#2}}{d {#1}}}
\newcommand{\ODat}[2]{\left. \frac{d}{ d{#1} } \right|_{{#1}={#2}}}
\newcommand{\ODfat}[3]{\left. \frac{d{#3} }{ d{#1} } \right|_{{#1}={#2}}}

% sums
\newcommand{\sumn}{\sum_{n=1}^\infty}
\newcommand{\summ}{\sum_{m=1}^\infty}
\newcommand{\sumk}{\sum_{k=1}^\infty}
\newcommand{\sumi}{\sum_{i=1}^\infty}
\newcommand{\sumj}{\sum_{j=1}^\infty}

\newcommand{\sumnz}{\sum_{n=0}^\infty}
\newcommand{\summz}{\sum_{m=0}^\infty}
\newcommand{\sumkz}{\sum_{k=0}^\infty}
\newcommand{\sumiz}{\sum_{i=0}^\infty}
\newcommand{\sumjz}{\sum_{j=0}^\infty}

\newcommand{\sumik}{\sum_{i=1}^k}
\newcommand{\sumjk}{\sum_{j=1}^k}
\newcommand{\summk}{\sum_{m=1}^k}

\newcommand{\sumin}{\sum_{i=1}^n}
\newcommand{\sumjn}{\sum_{j=1}^n}
\newcommand{\sumkn}{\sum_{k=1}^n}
\newcommand{\summn}{\sum_{m=1}^n}

\newcommand{\sumim}{\sum_{i=1}^m}
\newcommand{\sumjm}{\sum_{j=1}^m}
\newcommand{\sumkm}{\sum_{k=1}^m}
\newcommand{\sumnm}{\sum_{n=1}^m}

\newcommand{\sumnN}{\sum_{n=1}^N}
\newcommand{\sumiN}{\sum_{i=1}^N}
\newcommand{\sumjN}{\sum_{j=1}^N}
\newcommand{\sumkN}{\sum_{k=1}^N}
\newcommand{\summN}{\sum_{m=1}^N}

\newcommand{\sumizk}{\sum_{i=0}^k}
\newcommand{\sumjzk}{\sum_{j=0}^k}
\newcommand{\summzk}{\sum_{m=0}^k}

\newcommand{\sumizn}{\sum_{i=0}^n}
\newcommand{\sumjzn}{\sum_{j=0}^n}
\newcommand{\sumkzn}{\sum_{k=0}^n}
\newcommand{\summzn}{\sum_{m=0}^n}

\newcommand{\sumizm}{\sum_{i=0}^m}
\newcommand{\sumjzm}{\sum_{j=0}^m}
\newcommand{\sumkzm}{\sum_{k=0}^m}
\newcommand{\sumnzm}{\sum_{n=0}^m}

\newcommand{\sumnzN}{\sum_{n=0}^N}
\newcommand{\sumizN}{\sum_{i=0}^N}
\newcommand{\sumjzN}{\sum_{j=0}^N}
\newcommand{\sumkzN}{\sum_{k=0}^N}
\newcommand{\summzN}{\sum_{m=0}^N}

\newcommand{\sumijn}{\sum_{i,j=1}^n}

% convergence
\newcommand{\xraN}{\xra{N\to\infty}}
\newcommand{\xran}{\xra{n\to\infty}}
\newcommand{\xram}{\xra{m\to\infty}}
\newcommand{\xrak}{\xra{k\to\infty}}
\newcommand{\xraj}{\xra{j\to\infty}}
\newcommand{\xral}{\xra{l\to\infty}}
\newcommand{\xraR}{\xra{R\to\infty}}
\newcommand{\xrar}{\xra{r\to0}}
\newcommand{\xraeps}{\xra{\epsilon \to 0}}


% limits
\newcommand{\limn}{\lim_{n\to\infty}}
\newcommand{\limm}{\lim_{m\to\infty}}
\newcommand{\limN}{\lim_{N\to\infty}}
\newcommand{\limez}{\lim_{\epsilon\to0}}
\newcommand{\limhz}{\lim_{|h|\to0}}
\newcommand{\limj}{\lim_{j\to\infty}}
\newcommand{\limi}{\lim_{i\to\infty}}
\newcommand{\limk}{\lim_{k\to\infty}}
\newcommand{\limh}{\lim_{h\to\infty}}

% limsup
\newcommand{\limsupn}{\limsup_{n\to\infty}}
\newcommand{\limsupk}{\limsup_{k\to\infty}}
\newcommand{\limsuph}{\limsup_{h\to\infty}}

% liminf
\newcommand{\liminfn}{\liminf_{n\to\infty}}
\newcommand{\liminfk}{\liminf_{k\to\infty}}
\newcommand{\liminfh}{\liminf_{h\to\infty}}

% integrals
\newcommand{\intii}{\int_{-\infty}^{\infty} }
\newcommand{\intzi}{\int_{0}^{\infty}}
\newcommand{\intiz}{\int_{-\infty}^0}
\newcommand{\intoi}{\int_{1}^{\infty}}
\newcommand{\intzo}{\int_{0}^{1} }

% integral symbols
\def\Xint#1{\mathchoice
{\XXint\displaystyle\textstyle{#1}}%
{\XXint\textstyle\scriptstyle{#1}}%
{\XXint\scriptstyle\scriptscriptstyle{#1}}%
{\XXint\scriptscriptstyle\scriptscriptstyle{#1}}%
\!\int}
\def\XXint#1#2#3{{\setbox0=\hbox{$#1{#2#3}{\int}$ }
\vcenter{\hbox{$#2#3$ }}\kern-.58\wd0}}
\def\ddashint{\Xint=}
\def\avint{\Xint-} % average intetegral

% for use in integrals
\newcommand{\dd}{\, d}
\newcommand{\dx}{\dd x}
\newcommand{\dt}{\dd t}
\newcommand{\dds}{\dd s}
\newcommand{\du}{\dd u}
\newcommand{\dm}{\dd m}
\newcommand{\dlam}{\dd \lambda}
\newcommand{\dmu}{\dd \mu}
\newcommand{\dnu}{\dd \nu}
\newcommand{\dpi}{\dd \pi}
\newcommand{\dgam}{\dd \gamma}
\newcommand{\dz}{\dd z}
\newcommand{\dy}{\dd y}
\newcommand{\dr}{\dd r}
\newcommand{\dR}{\dd R}
\newcommand{\dw}{\dd w}
\newcommand{\dk}{\dd k}
\newcommand{\dA}{\dd A}
\newcommand{\dv}{\dd v}
\newcommand{\dP}{\dd \PP}
\newcommand{\dS}{\dd S}
\newcommand{\dtheta}{\dd \theta}
\newcommand{\dphi}{\dd \phi}
\newcommand{\drho}{\dd \rho}
\newcommand{\dxi}{\dd \xi}
\newcommand{\dzeta}{\dd \zeta}
\newcommand{\dpsi}{\dd \psi}
\newcommand{\deta}{\dd \eta}
\newcommand{\dXi}{\dd \Xi}
\newcommand{\ddp}{\dd p}
\newcommand{\dq}{\dd q}
%
% modified from Peter Constantin
%\newcommand{\be}{\begin{equation}}
%\newcommand{\ee}{\end{equation}}
%\newcommand{\beu}{\begin{equation*}}
%\newcommand{\eeu}{\end{equation*}}
%\newcommand{\ba}{\begin{array}{l}}
%\newcommand{\ea}{\end{array}}
%\def\bal#1\eal{\begin{align}#1\end{align}}
%\def\balu#1\ealu{\begin{align*}#1\end{align*}}
%\def\bali#1\eali{\begin{equation}\begin{aligned}#1\end{aligned}\end{equation}}
%\newcommand{\bpde}{\begin{pde}}
%\newcommand{\epde}{\end{pde}}
%\newcommand{\bpdeu}{\begin{pde*}}
%\newcommand{\epdeu}{\end{pde*}}
%\newcommand{\pa}{\partial}
%\newcommand{\na}{\nabla} 
%\newcommand{\la}{\label}
\newcommand{\fr}{\frac}
%\newcommand{\pad}[2]{\frac{\partial {#1}}{\partial {#2}}}
\newcommand{\be}{\begin{equation}}
\newcommand{\ee}{\end{equation}}
\newcommand{\beu}{\begin{equation*}}
\newcommand{\eeu}{\end{equation*}}
\newcommand{\ba}{\begin{array}{l}}
\newcommand{\ea}{\end{array}}
\def\bal#1\eal{\begin{align}#1\end{align}}
\def\balu#1\ealu{\begin{align*}#1\end{align*}}
\def\bali#1\eali{\begin{equation}\begin{aligned}#1\end{aligned}\end{equation}}
\newcommand{\pad}[2]{\frac{\partial {#1}}{\partial {#2}}}

% nonmath
\newcommand{\tbf}{\textbf}
\newcommand{\tit}{\textit}

%% For use with breqn package
\newcommand{\MC}{\condition*}% math condition
\newcommand{\TC}{\condition}% text condition

%% My own version of the above
\newcommand{\ATC}[1]{\qquad\text{#1}}

%% Environments
\newenvironment{pde}{\begin{equation}\left\{ \begin{alignedat}{2}}{\end{alignedat} \right. \end{equation}}
\newenvironment{pde*}{\begin{equation*}\left\{ \begin{alignedat}{2}}{\end{alignedat}\right. \end{equation*}}
% Example:
% \begin{pde*}
% -\Delta u &= 0 &\quad& \text{in $U$} \\
% u &= 0 & &\text{on $\bd U$}
% \end{pde*}


%% Redefining stuff...
%\let\divsym\div
%\renewcommand{\div}{\Div} % divergence, not division symbol! (who needs the division symbol?!)
\let\ophi\phi
\renewcommand{\phi}{\varphi} % \varphi is prettier; use \ophi to get the standard latex phi
%\renewcommand{\mid}{:}





\usepackage{mathabx}

% extra packages used in this doc
\usepackage{bm} % extra bold for greek letters
\newcommand{\mli}[1]{\mathit{#1}} %multiple letters in math
%\newcommand{\varA}[1]{{\operatorname{#1}}}
\newcommand{\var}[1]{{\operatorname{\mathit{#1}}}} %dash in math with italics

\usepackage{booktabs, stackengine}
\usepackage{array}
\newcolumntype{L}[1]{>{\raggedright\arraybackslash}p{#1}}
\setstackEOL{\#}
\setstackgap{L}{12pt}
  
% Page setup
\usepackage[letterpaper,margin=1in]{geometry}
\usepackage[doublespacing]{setspace} % double spacing
\usepackage{tocloft}
\usepackage[figlist, tablist,markers]{endfloat} % put tables, figures at the end 


% wrap text in a table
%\renewcommand{\arraystretch}{1.5}
\usepackage{tabulary}
\usepackage[newcommands]{ragged2e}

% under bar
\usepackage{etoolbox}
%\usepackage{fixltx2e}
\usepackage{accents}
\robustify{\underaccent}
\newcommand{\ubar}[1]{\underaccent{\bar}{#1}}  
%\DeclareRobustCommand{\ubar}[1]{\underaccent{\bar}{#1}}

%% Theorems, etc.
\theoremstyle{definition}
\newtheorem{defn}{Definition}
\newtheorem{exmp}[defn]{Example}
\newtheorem{claim}{Claim}
\newtheorem{claimproof}{Proof of claim}

\theoremstyle{plain}
\newtheorem{thm}[defn]{Theorem}
\newtheorem{prop}[defn]{Proposition}
\newtheorem{lem}[defn]{Lemma}
\newtheorem{cor}[defn]{Corollary}
\newtheorem{fact}[defn]{Fact}

\theoremstyle{remark}
\newtheorem{rem}[defn]{Remark}

%Todo notes
%\usepackage[obeyFinal]{todonotes}
\usepackage{todonotes}

% appendix 
\usepackage[toc,page]{appendix} % appendix

% images
\usepackage{graphicx}
\usepackage{caption}
\usepackage{subcaption}
\usepackage[capposition=top]{floatrow} % notes for figure
\usepackage{ragged2e}
  
% tables
\usepackage{booktabs, caption, fixltx2e, subcaption, siunitx}
\usepackage[flushleft]{threeparttable}
%\usepackage{array}
%\newcolumntype{L}[1]{>{\raggedright\arraybackslash}p{#1}}
%\newcolumntype{C}{@{\extracolsep{3cm}}c@{\extracolsep{0pt}}}%
%\usepackage{multirow}
%\newcommand{\rowgroup}[1]{\hspace{-1em}#1}
%\newcommand{\sym}[1]{\rlap{$#1$}} % for symbols in Table
\usepackage{longtable} % See help in http://www2.astro.psu.edu/gradinfo/psuthesis/longtable.html
\usepackage{pdflscape}
\usepackage{afterpage}
\usepackage{capt-of}% or use the larger `caption` package
\usepackage{adjustbox}
\usepackage{rotating} % rotate table for sidways

% single-spaced in tables
%\usepackage{etoolbox} 
%\AtBeginEnvironment{tabular}{\singlespacing}
%\AtBeginEnvironment{tablenotes}{\singlespacing}
%\AtBeginEnvironment{tabular*}{\singlespacing}
%\AtBeginEnvironment{threeparttable}{\singlespacing}
%\AtBeginEnvironment{longtable}{\singlespacing}

% wrap text in a table
%\renewcommand{\arraystretch}{1.5}
\usepackage{tabulary}
\usepackage[newcommands]{ragged2e}


% Sections
%\makeatletter
%\renewcommand{\section}{\@startsection
%{section}%                   % the name
%{1}%                         % the level
%{0mm}%                       % the indent
%{0.5\baselineskip}%            % the before skip
%{0.5\baselineskip}%          % the after skip
%{\normalfont\bf}} % the style
%\renewcommand{\subsection}{\@startsection
%{subsection}%                   % the name
%{1}%                         % the level
%{0mm}%                       % the indent
%{0.5\baselineskip}%            % the before skip
%{0.5\baselineskip}%          % the after skip
%{\normalfont\bf}} % the style
%\makeatother
%\renewcommand{\refname}{\normalsize References}

% PDF links, etc.
\usepackage[pdftex]{hyperref}
\hypersetup{
	colorlinks=true,
	urlcolor=blue,
	linkcolor=blue,
	citecolor=blue,
	linktoc=all,
	pdfauthor='Kunhee Kim'
}


% References
\usepackage{natbib}
\usepackage{cite}

% title page
\usepackage{titling} % title abstract on the same page

% footnote without markers
\newcommand\blfootnote[1]{%
  \begingroup
  \renewcommand\thefootnote{}\footnote{#1}%
  \addtocounter{footnote}{-1}%
  \endgroup
}


\newcommand{\myreferences}{/Users/kimk13/Dropbox/Research/dissertn_bib}



\title{The Effect of Workforce Assignment on Performance: Evidence from Home Health Care}

\author{Guy David\thanks{The Wharton School, University of Pennsylvania, Philadelphia, PA 19104, USA.} \quad \quad Kunhee Lucy Kim\thanks{New York University School of Medicine, New York, NY 10016, USA.}}
% \footnotemark[1]}


\title{The Effect of Workforce Assignment on Performance: Evidence from Home Health Care}
\date{\parbox{\linewidth}{\centering%
	February 2018 \hspace*{0cm} \endgraf\medskip
}}

\begin{document}

\begin{singlespace}
\maketitle
\thispagestyle{empty}


\begin{abstract}
Effective workforce assignment has the potential for improving performance. Using novel home health data combining provider work logs, personnel data, and detailed patient records, we estimate the effect of provider handoffs---a marker of care discontinuity---on hospital readmissions, an important performance measure for healthcare systems. We use workflow interruption caused by attrition and providers' work inactivity as an instrument for nurse handoffs.
We find handoffs to substantially increase hospital readmissions. Our estimates imply that a single handoff increases the likelihood of 30-day hospital readmission by 16 percent and one in four hospitalizations during home health care would be avoided if handoffs were eliminated. Moreover, handoffs are more detrimental for high-severity patients and expedite hospital readmission. The frequency and sequencing of handoffs also affect the likelihood of rehospitalization.
\blfootnote{
\noindent
We thank David Grabowski, Jonathan Gruber, Pierre Thomas L\'{e}ger, Mark Pauly, and participants of the 2016 American Society of Health Economists meeting and Annual Health Economics Conference.
We also thank David Baiada, Alan Wright, Ann Gallagher, and Stephanie Finnel for tremendous insight and data support throughout the project. The Leonard Davis Institute of Health Economics provided financial support.}
\end{abstract}


\end{singlespace}


\newpage
\section{Introduction}


Workforce allocation and scheduling are routinely designed to achieve multiple organizational goals, with efficiency typically viewed as the leading objective. Efficient workforce assignment entails the matching of task and talent \citep{Garicano2004}, management of planned and unplanned absences \citep{Ehrenberg1970, Allen1983}, exigency and geographical optimization, and responsiveness to demand shocks \citep{Hamermesh1996}.
Beyond efficiency, workforce allocation goals may include rewarding seniority, promoting workforce equity, and enabling effective learning and synergy \citep{Mas2009}. These goals potentially compromise short-term efficiency but at the same time raise employee satisfaction and reduce costly turnover. Another set of objectives is linked with the use of workforce assignment to achieve higher quality. While often in conflict with cost minimization goals, higher quality may be rewarded directly through higher willingness to pay and indirectly through increased reputation.



Hospitals and health care systems implement strategies to improve the quality of care for all patients through focusing on patient safety, reducing medical errors, establishing evidence-based guidelines, and lowering the rate of unnecessary and preventable intervention \citep{Kozak2001, Makary2016}.
In fact, ensuring continuity of care within and across care settings is identified as a pillar of quality improvement \citep{Richardson2001}.\footnote{Continuity of care has also been shown to reduce utilization and costs of care \citep{Raddish1999}, such as by reducing the number of emergency department visits and shortening the length of hospital stays \citep{Wasson1984}.}
Continuity of care across settings involves, by definition, a multi-professional pathway that emphasizes the need for care coordination. On the other hand, continuity of care within a setting is achieved by workforce allocation, and in particular a continuous relationship between a patient and a single health care professional who is the sole source of care and information for the patient.

However, the achievement of continuity of care requires costly deployment of resources. Ensuring smooth transitions in care and effective transmission of information between providers likely imposes massive constraints that interfere with the goal of optimizing scheduling to minimize workforce turnover and contractual disruptions. Thus, efficient workforce assignment may lead to reductions in quality of care through compromised care continuity. Using a novel data set from a large multi-state freestanding home health agency, this paper quantifies the effect of within-setting care discontinuity caused by workforce assignment on hospital readmissions, a common quality of care marker.

Spending due to unplanned hospital readmissions was estimated at \$17.4-\$25 billion annually, which would translate to 16-22\% of the total Medicare spending on inpatient hospital services \citep{PricewaterhouseCooper2008, Jencks2009}.
The national all-cause potentially preventable readmission rates for this population was 11\% in 2014 \citep{MedPAC2016medicare}.
Starting in October 2012, the Center for Medicare and Medicaid Services (CMS) lowered its payment to hospitals with excess readmissions over the national average by up to 3\%.\footnote{The amount of reduction in payment was up to 1 percent in FY 2013, the first year of the penalty (so-called the Hospital Readmissions Reduction Program), and up to 2 percent in FY 2014.}
Facing financial penalties, hospitals use management strategies and modifications to their organizational structure to prevent hospital readmissions.
For example, hospitals vertically integrated with post-acute care providers such as home health agencies (HHAs) to improve post-discharge care coordination, as increased reliance on home health has been shown to be associated with a reduction in hospital readmissions \citep{Polsky2014}.\footnote{\citet{Naylor1999} discuss hospitals that instituted programs to provide patient education before discharge, increased patient follow-up, and expanded the use of health information technology to track readmissions and integrate care across settings; \citet{Kim2015} show that admitting ER patients to the Intensive Care Units could substantially reduce hospital readmissions, and therefore suggest implementing admission criteria based on objective measures of patient risk as well as physicians' discretionary information as a promising way to decrease hospital readmissions.
}
Moreover, hospitals rely on post-acute care entities to reduce avoidable readmissions \citep{Naylor2012}. Once patients are discharged from hospitals, post-acute care providers monitor and treat still frail patients over an extended period of time.\footnote{In the case of home health care, the default length of an episode is 60 days for Medicare patients.
}
Thus, post-acute care providers can impact the frequency of hospital readmissions by implementing workforce assignment strategies that promote care continuity.








We focus on home health care as it is an important and rapidly growing segment of the health care delivery system.
Over the past decade, payment for home health services more than doubled \citep{MedPAC2016hh}.
This rapid growth may be attributed to its appeal to patients who prefer to recover at home, providers who prefer to shorten hospitalization lengths, and insurers who benefit from cheaper care at home than care in brick-and-mortar institutions. Home health care is recognized as a partial substitute for institutional long-term care \citep{Guo2015}. The importance of home health care has also increased with the rise of enhanced care coordination and shared savings models such as Accountable Care Organizations or Bundled Payments for Care Improvement \citep{Sood2011}.

Studying the intricacies of home health care provision and its impact on hospital readmissions is timely and important. Before the ACA, there was no competitive pressure for HHAs and no financial incentives to reduce readmissions, with three in ten post-acute home health stays resulting in a hospital readmission among Medicare patients \citep{MedPAC2014hh}.\footnote{This figure could also be attributed to the fact that patients being discharged to home health care tend to be sicker and at a higher risk of hospital readmissions that those being discharged to home.
}
However, with readmission penalties and the emphasis on population health management, home health has become a way to allow for continuity of care outside of the hospital and effectively manage the patient health to prevent unnecessary readmissions.
Freestanding agencies often view the ability to mitigate hospital readmission as a key competitive differentiator in contracting with hospitals \citep{Worth2014}.
Therefore, it is important to uncover potential mechanisms that lead to better care continuity and patient outcomes.

In this paper, we use novel data containing over 43,000 home health patient episodes and spanning 89 autonomously run home health offices in 16 states. The data provide detailed information, which includes visit logs for all Medicare patients, work logs and human resources data for all home health providers, as well as all patient demographic and health risks collected as part of the Outcome and Assessment Information Set (OASIS) required by the CMS.
In addition, our data are linked with individual patients' hospital readmissions.
We measure care discontinuity by handoffs between skilled nurses over a patient's episode of care, which are immediately affected by offices' workforce allocation decisions.\footnote{Skilled nurses refer to registered nurses (RNs) or licensed practical nurses (LPNs).
}
We estimate a plausibly causal effect of provider handoffs on hospital readmissions using day-to-day human resources data on providers' absence, assignment to an alternative office, and job termination to instrument for handoffs.
Unplanned employee absences in the US health care and social assistance sector consumed 1.9\% of all scheduled work hours in 2016 \citep{BLS2017}.
To uncover the mechanisms underlying the effect of handoffs, we also examine whether handoffs affect hospital readmissions differently by underlying patient severity and by the frequency and sequencing of handoffs, respectively, and whether handoffs affect time to readmission.


Estimating the effect of handoffs in home health care on the probability of readmissions raises endogeneity concerns.
While we observe a great deal of patient characteristics as well as labor supply conditions, the data do not provide us with the actual care plan for each patient's episode of care. The care plan is plausibly linked to unobserved patient severity and hence to the risk of hospital readmissions. As we discuss in the paper, it is challenging to determine the sign of the omitted variable bias caused by unobserved patient characteristics.
To address this endogeneity problem, we use detailed provider-day level data on nurses' availability to instrument for both handoffs and the probability of receiving a visit. The identification assumption is that skilled nurses' absence affects rehospitalization only through its effect on care discontinuity either through missed visits or handoffs. In addition, and as explained in greater detail in our methods section, we control for the dynamic changes in patients' health status during a home health episode by limiting the variation in our data to reflect the number of days since the last nurse visit as well as supply and demand characteristics at the nurse- and office-day level. Together with the patient's initial health assessment, these controls help mitigating potential confounding effects.

Using the cross-sectional variation,
we find that patients experiencing nurse handoffs are 24\% more likely to be readmitted to a hospital.
This estimate more than doubles in magnitude when we use the instrumental variables (IV) method.
Our results are robust to controlling for days since last visit as well as a rich set of patients' health risk, demographic, and comorbidity factors, office fixed effects, time fixed effects, and home health day fixed effects.
Controlling for home health day fixed effects is especially important because the probabilities of hospital readmissions and handoffs rapidly decline over the course of a home health episode.
Furthermore, in our analysis of potential mechanisms, we find that handoffs are more detrimental for high-severity patients and expedite hospital readmission. The frequency and sequencing of handoffs also affect the likelihood of rehospitalization with the first handoffs having the strongest effect on increasing hospital readmissions.



A number of potential mechanisms may account for the effect of provider handoffs on hospital readmissions.
First, information transmission between providers involved in a handoff may be incomplete and lead to potentially inappropriate care \citep{Riesenberg2009}.
Second, holding the number of visits constant, handoffs lower the time spent with each individual provider, and hence depreciates the relationship stock built between providers and patients, which has been shown to improve patient outcomes \citep{Saultz2005}.
Third, repeated visits enhance the development of patient-specific knowledge, which has limited applicability to other patients. Therefore, patients experiencing a handoff lose access to providers most familiar with their case. For example, previous literature emphasizes the development of firm- or patient-specific skills among cardiac surgeons and radiologists, which are associated with a reduction in patient mortality rates \citep{Huckman2006}.
The three channels above serve as theoretical underpinning for our findings of a positive link between provider handoffs and hospital readmissions.




Contributing to the literature on the impact of nursing attributes on patient health outcomes \citep{Aiken2002, Bae2010, Needleman2011, Cook2012, Lin2014, Lu2016, Hockenberry2016}, this paper provides the first set of results linking workforce assignment decisions in the post-acute care setting to hospital readmissions.
Most of the literature on care continuity has focused on transitions of care across settings, especially on care transitions from hospitals to post-acute care facilities \citep{Naylor1999}.
Work on continuity of care within a setting has focused almost exclusively on patient handoffs in shift-based environments, which are shown to be associated with low quality of care marked byß slowdown in service delivery, medical and surgical errors, malpractice cases with communication problems, and (potentially preventable) adverse patient outcomes \citep{Laine1993, Petersen1994, Riesenberg2009}.
However, the external validity of results in a shift-based environment may be weak when considering non-shift-based environments, such as home health.  Shift-based handoffs are inevitable due to a trade-off between the length of a shift and the number of handoffs. When physician or nurse shifts are lengthened, a patient is more likely to see the same provider during the course of treatment. At the same time, a longer shift would increase provider fatigue and the risk of making mistakes, especially, towards the end of long shifts \citep{Brachet2012}. In contrast, in home health care, handoffs are largely avoidable through coordinated scheduling given that providers typically visit patients with several days in between. In our data, 38\% of patients are seen consistently by a single nurse throughout their episode of care. Hence, zero handoffs are frequent in a non-shift-based environment.  And yet, prioritizing continuity of care may be costly in that it may come at the expense of flexibility in scheduling, employee satisfaction, and ultimately retention.
Therefore, quantifying the effect of discontinuous home health care has important implications for the use of workforce assignment in improving quality of care, and provides a currency to assess the importance of prioritizing care continuity over other goals typically achieved through schedule architecture.




The outline of the article is as follows. In Section~\ref{sec:databig}, we describe the data and present our measures of care discontinuity. In Section~\ref{sec:empr_strategy}, we discuss our identification strategy. In Section~\ref{sec:results}, we discuss our baseline empirical results as well as our IV estimation results. In Section~\ref{sec:mechanism}, we explore the potential mechanisms underlying the relationships between handoffs and rehospitalization. Section~\ref{sec:conclusion} concludes the paper.




















\section{Data} \label{sec:databig}

\subsection{Data and Summary Statistics} \label{sec:data_ch3}
This paper uses a novel and rich data set of home health visits, patient health status assessment, and provider work logs as well as indicators for patient hospital readmissions.
We obtained data on all home health stays for Medicare patients from a large for-profit freestanding home health company, which provides home health care services in 89 offices in 16 states.\footnote{These offices are located in 16 states: Arizona, Colorado, Connecticut, Delaware, Florida, Hawaii, Massachusetts, Maryland, North Carolina, New Jersey, New Mexico, Oklahoma, Pennsylvania, Rhode Island, Virginia, Vermont.
}\footnote{\citet{David2013} show that in vertically integrated HHAs owned by hospitals, post-acute care patients are admitted to HHAs in earlier stages of recovery without a significant difference in readmission rates.
}
Since each office autonomously decides scheduling and staffing and is run as a profit center, we can regard each office as a separate hiring and contracting unit in our empirical analysis.
This large set of independently run offices alleviates some concern about the generalizability of our results to other HHAs even if they all belong to one company.\footnote{During 2013, compared to a national sample of freestanding agencies, home health offices in our sample tend to be larger, have a lower share of visits provided for skilled nursing and instead have a higher share of visits provided for therapy, and have a lower share of episodes provided to dual-eligible Medicare or Medicaid beneficiaries, which seem to be more common characteristics of proprietary agencies \citep{Cabin2014, MedPAC2016hh}.
 However, home health offices in our data provide a similar total number of visits per episode and serve a similar age group on average.
}
Our sample period covers 44 months between January 2012 and August 2015, for which we have full patient, worker, and office data.

Our main outcome is hospital readmission, which is an important marker of quality and can potentially lead to financial penalties for hospitals. Rehospitalizations among patients receiving home health is common. Our analysis focuses on Medicare patients, who comprise a majority of home health patients and have a high probability of hospital readmissions.\footnote{In the data, on average, 69\% of home health episodes in each month are paid for by Medicare (including all Medicare FFS, private Medicare Advantage, and Medicare Part B) either as a primary or secondary payer across 93 offices and months during the sample period.
Nationally, Medicare patients had 29\% readmission rates among post-hospital home health stays \citep{MedPAC2014hh}.
}
Furthermore, our analysis focuses on care discontinuity for skilled nursing because most visits are for skilled nursing care and it provides most medically relevant service that could potentially determine the likelihood of hospitalization \citep{Russell2011}.\footnote{In addition to nurses, there are typically additional providers who visit patients during a home health episode. Those include home health aides, physical therapists, speech-language pathologists, occupational therapists, and medical social services workers.
}

Our patient data are provided at the patient visit level as well as the patient episode-admission level.\footnote{Medicare FFS pays a prospective payment for each 60-day episode. For patients requiring more care, episodes may be extended by another 60 days during a given home health admission.
}
To construct our patient-day level data, we merge the patient episode level data with the visit level data. Home health episodes can end by either a discharge or a hospitalization.
The exact dates of these end points for each episode are obtained from the home health admission level OASIS data. These data also provide a rich set of health risk factors.

We also obtained human resources data containing work logs for all providers' visits.
We merge the patient-day level data with provider-day level work log data to identify handoffs and link them with hospital readmissions.
Separately, we also use the provider-day level work logs to construct instruments of providers' inactivity statuses, as described in Section~\ref{sec:availability}.

Finally, we construct office-day level data spanning all 89 offices. This data set tracks ongoing episodes and all nurses in each office providing services on each day. This is then merged with the patient-day level data to provide office-level demand and supply conditions.

To construct our final patient-day level sample for analyses, we exclude patients who had multiple subsequent home health episodes as these home health stays may have different patterns of visit schedules and provider handoffs.\footnote{More precisely, enrolling patients into subsequent episodes has been shown to exhibit a degree of strategic behavior. For example, after the introduction of the home health prospective payment system in 2000, agencies increased the number of episodes per patient \citep{Kim2015a}.}
Since our measure of care discontinuity---handoffs---occur across visits, we exclude episodes consisting of a single visit.
Finally,
we restrict to home health episodes with a prior hospitalization in the past 14 days.\footnote{Home health admissions preceded by a hospital stay account for 35.5\% of all Medicare home health admissions in our sample.
}
Our final sample includes 43,740 unique home health episodes and 1,031,904 patient days under home health.

Table~\ref{tab:summstats} reports the summary statistics at different levels of aggregation: Panel A at the office-day level; Panel B at the patient-episode level; and Panel C at the patient episode-day level. In our sample, 16.6\% of episodes involve a hospital readmission, and most of the readmissions occur within 30 days of hospital discharge.\footnote{Another outcome reported in the OASIS survey is death at home. We do not use it as an outcome because it is rare. Out of more than one million patient-days, death occurs at a rate of 0.18\%.}
The average home health episode in our sample involved 6 nurse visits over a period of 33 days, with 87\% of home health episodes involving between 3 and 12 nurse visits.


Figure~\ref{fig:pct_rehosp} presents the number of ongoing episodes and number of readmissions occurring by home health day.
It suggests that both the probabilities of being under home health care and readmission decline with home health days. Thus, we control for day of home health fixed effects as well as the number of visits and days since last visit to examine the effect of discontinuous care on the probability of rehospitalization across patients with identical number of visits, spacing of visits, and episode length. We discuss this further in Section~\ref{sec:spec}.


\subsection{Measuring Care Discontinuity: Provider Handoffs}\label{sec:measure_handoff}

We focus on the notion of provider handoffs to measure care discontinuity. Many studies on care discontinuity focused on shift-based settings in which there is a salient trade-off between the length of shifts and quality of care \citep{Laine1993, Petersen1994, Riesenberg2009, Brachet2012}.
By making the shifts longer, you can provide more continuous care but this comes at the expense of providers' fatigue towards the end of the shifts. In contrast, in home health care settings, handoffs can plausibly be eliminated since 24/7 coverage is rare and visits are typically provided with several days in between.
In our sample, the average number of days between visits is 5.
Thus, continuous home health care can be naturally conceptualized as seeing the same provider repeatedly, and discontinuous care as a break in it---a provider handoff.
Handoffs can also capture a disruption in important aspects of care continuity---uninterrupted service delivery and trusting relationship between service provider and client or caregiver---emphasized by the key stakeholders in the home health industry \citep{Woodward2004}.

For the estimation, we define a ``handoff state'' as a series of days, beginning on the day a visit by a different skilled nurse occurs and ending on one day before the day that the same skilled nurse visits again (i.e. when continuity of care is restored). Put differently, for each patient $i$ and day $t$, an indicator of having a nurse handoff equals $1$ if $i$'s last nurse is different from the nurse who cared for $i$ in the preceding visit; and $0$ otherwise.\footnote{Under this definition of handoffs, a patient could be in a handoff state even on days she has no nurse visit if those days follow a visit during which an actual handoff occurred.  We choose this definition because we view that a patient is ``at risk'' of being readmitted to a hospital after a handoff occurs.
Our results are robust to restricting the sample only to visit days (results are available upon request).
}
Under this definition of handoffs, only 38\% of patient episodes experience no handoff during the episode of care, with the remaining 62\% of patient episodes having at least on handoff.

Figure~\ref{fig:fr_cho_gt} tracks additional variants of handoffs across the home health episode's length.
Figure~\ref{fig:fr_cho_gt} links home health day with the fraction of patient episodes with at least one, two, three or four handoffs.
By the 10th home health day, the fraction of episodes with at least one handoff is 54\%, 18\% with at least two handoffs, 4\% with at least  three handoffs, and 1\% with at least four handoffs.
In comparison, by the 30th home health day, the fraction of episodes with at least one handoff is 64\%, 39\% with at least two handoffs, 21\% with at least three handoffs, and 11\% with at least four handoffs.
Similarly, Figure~\ref{fig:frv_ho} shows the fraction of patient-days with nurse handoffs conditional on having a nurse visit.
Again, handoffs are substantially more likely to occur early in the home health episode and then sharply decline with more home health days.



















\section{Empirical Strategy} \label{sec:empr_strategy}

\subsection{Baseline Specification} \label{sec:spec}

We estimate linear probability models with the following specification:
\begin{equation}\label{eq:regeq}
Readmit_{ikt} = \alpha + \beta H_{ikt} + \gamma V_{ikt} + \delta_1 X_{ikt} + \delta_2 P_{ik} + \delta_3 W_{kt} + \delta_4 D_t + \theta_k + \epsilon_{ikt}
\end{equation}
where $Readmit_{ikt}$ is an indicator variable for whether patient $i$ served by office $k$ is readmitted to a hospital on day $t$;
$H_{ikt}$ is an indicator variable for handoffs described in Section~\ref{sec:measure_handoff};
$V_{ikt}$ is an indicator variable for having a nurse visit;
$X_{ikt}$ is a vector of patient-office-day level variables;
$P_{ik}$ is a vector of patient-office-level variables;
$W_{kt}$ is a vector of office-day level variables;
$D_t$ is a day-level variables;
$\theta_k$ is office fixed effects.

Whether a patient is readmitted to a hospital may depend on the progression of her severity over the course of home health care, making it important to control for dynamic changes in the patient's daily health status.
Beyond an initial assessment in the first visit, home health agencies do not systematically measure and collect data on the patient's health status in subsequent visits.
Therefore, we cannot directly control for the dynamic changes in patients' health.
However, we use a number of dynamic proxies of patients' real-time health status.
First, we control for whether a patient has a nurse visit on a given day, $V_{ikt}$, as a sicker patient is more likely to have a nurse visit.
Second, we control for the number of days since last visit by a nurse in the vector $X_{ikt}$ together with the patient-level mean interval of days between consecutive nurse visits in the vector $P_{ik}$. The variation in the number of days since last nurse visit holding constant the expected frequency of visits during the episode could capture dynamic shifts in the patient's severity since a nurse's additional visit only after a short period of time may suggest that the patient has gotten sicker on that day. Third, for similar reasons, we include in the vector $X_{ikt}$ of patient-office-day level variables the number of days since last visit by any provider since the greater the gap between any home health visits, the more likely a patient is to have a readmission controlling for the average frequency of nurse visits.  Fourth, we control for the cumulative number of nurse visits provided to control for the effect of dynamic care intensity.\footnote{Our results are robust to excluding the number of days since last visit by any provider, the patient level mean interval of days between consecutive nurse visits, and the cumulative number of nurse visits provided. These results appear in Table~\ref{tab:iv_noendog} in Appendix~\ref{appendix:iv_noendog}.}\footnote{Additional potentially relevant variables to include in the vector $X_{ikt}$ are the cumulative number of unique nurses the patient has seen by home health day $t$ and the number of times each nurse has seen the patient. These variables capture an aspect of care disruption that is potentially more meaningful for patients experiencing a large number of handoffs. For example, a patient experiencing six handoffs may be cared for by six different nurses, or may experience multiple handoffs between the same two nurses. Our results are robust and even stronger when controlling for these two additional variables (these results are available upon request).}

In the vector $P_{ik}$ of patient level variables, we also include the following three groups of variables to adjust for underlying health risks of patients. First, a set of indicator variables associated with high risk of hospitalization, including history of 2 or more falls in the past 12 months, 2 or more hospitalizations in the past 6 months, a decline in mental, emotional, or behavioral status in the past 3 months, currently taking 5 or more medications, and others.\footnote{Over two thirds of patients were visited within 24 hours of hospital discharge (with 90\% of patients visited within 72 hours of hospital discharge. Similar results were obtained when we control for the number of days since hospital discharge. These results are reported in Table~\ref{tab:iv_days_fromhospdc} in Appendix~\ref{appendix:dist_days_fromhospdc}.}
Second, a set of indicator variables for patient demographics: age dummies for each age 66-94 and age 95 or higher (reference group is age 65), gender, race, insurance type, an indicator for having no informal care assistance available, and an indicator for living alone.\footnote{Insurance types include Medicare Advantage (MA) plans with a visit-based reimbursement, MA plans with an episode-based reimbursement, and dual eligible with Medicaid enrollment (reference group is Medicare FFS).
}
Third, a set of indicator variables for comorbidity factors, including indicators for 17 Charlson comorbidity index factors, indicators for overall health status, indicators for high-risk factors including alcohol dependency, drug dependency, smoking, obesity, and indicators for conditions prior to hospital stay within past 14 days including disruptive or socially inappropriate behavior, impaired decision making, indwelling or suprapublic catheter, intractable pain, serious memory loss and/or urinary incontinence.\footnote{Indicators for overall health status include indicators for very bad (patient has serious progressive conditions that could lead to death within a year), bad (patient is likely to remain in fragile health) and temporarily bad (temporary facing high health risks).
}

In the vector $W_{kt}$ of office-day level variables, we include the number of ongoing episodes and the number of skilled nurses working in the office-day to control for the time-variant caseload and labor supply conditions in each office.


The vector $D_t$ includes indicators for each home health day, indicators for each day of week, and indicators for each month-year.
The home health day fixed effects absorb any unobserved fixed characteristics of home health care depending on the timing within an episode, as illustrated in Figures~\ref{fig:pct_rehosp}, \ref{fig:fr_cho_gt}, and \ref{fig:frv_ho}.
We also control for month-year pairs as well as day of week indicators to control for any time-specific component of the variation in the likelihood of readmission such as lower probability of readmission in months with major holidays or weekends.
The office fixed effects $\theta_k$ absorb time-invariant office-specific or geographic differences in hospital readmissions, for example, through different hospital policies or state regulations concerning patient readmissions, such as states with Certificate-of-Need (CON) laws imposing home health entry restriction \citep{Polsky2014}.
Our estimates of the effects of handoffs would be based on the difference in the readmission rates between patients who experience a nurse handoff and those who do not on the same home health day, same month-year, day of week as well as in the same office after controlling for other observed characteristics discussed above.



\subsection{Identification Challenges} \label{sec:id_challenges}


Indicator variables for experiencing a handoff and having a nurse visit are endogenous due to the non-random provision of continuous care and nurse visits. To provide a plausibly causal estimate of the effect of handoffs on hospital readmissions, we need to use an exogenous measure of handoffs. To understand the identification strategy, rewrite the equation~\eqref{eq:regeq} in a more general form where the readmission outcome $Readmit_{ikt}$ is a function of handoffs $H_{ikt}$;
an indicator variable for having a nurse visit scheduled $V_{ikt}$;
other observable patient characteristics at the patient-office-day level $X_{ikt}$;
observable patient characteristics at the patient-office level $P_{ik}$;
office characteristics at the office-day level $W_{kt}$;
time fixed effects $D_t$;
unobservable patient characteristics on each day $U_{ikt}$;
and unobserved idiosyncratic component $\epsilon_{ikt}$ uncorrelated with $H_{ikt}$, $V_{ikt}$, $X_{ikt}$, $P_{ik}$, $W_{kt}$, $D_t$ or $U_{ikt}$:
\begin{equation}\label{eq:y}
Readmit_{ikt} = f(H_{ikt}, V_{ikt}, X_{ikt}, P_{ik}, W_{kt}, D_t, U_{ikt}, \epsilon_{ikt}).
\end{equation}

The identification assumption is that the likelihood of handoff varies only with observable patient characteristics and office- or provider-side daily characteristics, and is uncorrelated with unobservable patients' daily severity, i.e.
$$[H_{ikt}, V_{ikt} | X_{ikt}, P_{ik}, W_{kt}, D_t] \perp U_{ikt}.$$

Even though we control for a large number of patient, nurse, office and day characteristics as we described in Section~\ref{sec:spec}, we lack clear documentation of the care plan for each patient-episode of care. Consequently, we use an indicator variable for actual nurse visits, $\hat{V}_{ikt}$, as opposed to planned ones, $V_{ikt}$. However, whether a visit is actually provided is plausibly linked with unobserved patient severity and hence with the risk of hospital readmissions, thus resulting in $[H_{ikt}, \hat{V}_{ikt} | X_{ikt}, P_{ik}, W_{kt}, D_t] \notperp U_{ikt}.$


\graphicspath{ {/Users/kimk13/Dropbox/Wharton/Research/Labor/gph/anhandoff/} }
\begin{figure}[H]
\centering
\includegraphics[width=0.6\linewidth]{identification.png}
\footnotesize
\label{fig:identification}
\end{figure}
Put differently, it is difficult to determine the sign of the bias that omitting important patient characteristics will produce in the link between handoffs and hospital readmissions. The reason is that unobserved severity is potentially linked with handoffs and readmissions both directly and indirectly through care intensity. The direct link suggests that sicker patients are both more likely to experience adverse outcomes leading to hospital readmissions and less likely to experience handoffs since offices may try to provide more continuous care.
The indirect link is mediated by care intensity, that is, sicker patients will receive more frequent visits during their episode of care. Higher care intensity lowers the risk of hospital readmission but raises the likelihood of scheduling conflicts leading to handoffs. While the direct link suggests fewer handoffs and higher likelihood of hospital readmissions, the indirect link suggests the exact opposite.

\subsection{Identification Strategies} \label{sec:id_strategies}
As a first step to alleviate some of these concerns, we control for a rich set of patient-day variables that serve as a proxy for dynamic changes in unobserved patient-day level severity, as we explain in Section~\ref{sec:spec}.
These variables include the indicator for having a nurse visit and the number of days since last visit by a nurse or any other provider holding constant the mean frequency of nurse visits.
To further address the potential threat of endogeneity, we use detailed provider-day level data on breaks in nurses' availability to instrument for both handoffs and the probability of receiving a visit. The likelihood of having a nurse visit is also endogenous as only the actual visits provided are observed and whether a patient receives a nurse visit is correlated with unobservable daily severity level.
The instrument exploits the fact that skilled nurses' absences for various periods of time and reasons affect hospital readmission only through their effects on care discontinuity, that is either through missed visits or handoffs.

Call the vector of nurse availability breaks measures $B_{ikt}$.
Suppose that assumptions hold that (1) $B_{ikt}$ is strongly correlated with the endogenous variables $H_{ikt}$ and $\hat{V}_{ikt}$;
(2) $B_{ikt}$ is orthogonal to $U_{ikt}$ conditional on other vectors; and
(3) the observable vectors are separable from the last two unobserved vectors in $f(\cdot)$ in equation~\eqref{eq:y}.
We can identify the causal effect of provider handoffs on the likelihood of readmissions by estimating the system of equation~\eqref{eq:y} and
\begin{equation}\label{eq:H}
H_{ikt} = g( B_{ikt} , X_{ikt}, P_{ik}, W_{kt}, D_t, U_{ikt}, \eta_{ikt})
\end{equation}
 \begin{equation}\label{eq:V}
\hat{V}_{ikt} = h( B_{ikt} , X_{ikt}, P_{ik}, W_{kt}, D_t, U_{ikt}, \nu_{ikt})
\end{equation}
using the generalized method of moments, with the instrument moment condition
\begin{equation}
\E \bigg[ \Big\{ Readmit_{ikt} - f(H_{ikt}, \hat{V}_{ikt}, X_{ikt}, P_{ik}, W_{kt}, D_t ) \Big\} B_{ikt}  \bigg] = 0
\end{equation}
where $\eta_{ikt}$ and $\nu_{ikt}$ in equations~\eqref{eq:H} and \eqref{eq:V}, respectively, are idiosyncratic error terms for $H_{ikt}$ and $\hat{V}_{ikt}$ uncorrelated with $\epsilon_{ikt}$ and $U_{ikt}$.


\subsection{Breaks in Provider Availability} \label{sec:availability}
For the instrument set $B_{ikt}$ we use breaks in nurse availability on each day as a source of exogenous variation in the likelihood of having a nurse handoff and having a nurse visit.
By merging the provider-day level data with the patient-day level data, we track whether the  nurse seen by patient $i$ in the last visit is unavailable to serve $i$ today.
The last nurse can be assigned to one of six mutually exclusive states in each office $k$ on each day $t$:
(1) Active---visiting patients in $i$'s home office $k$;
(2) Short absence---not providing visits in any office for 1 to 6 consecutive days;
(3) Medium absence---not providing visits in any office for 7 to 14 consecutive days;
(4) Long absence---not providing visits in any office for 15 to 90 consecutive days;
(5) Assigned to other office---providing visits exclusively in a different office; and
(6) Attrition---not providing visits in any office for 91 or more consecutive days, or having the employment contract terminated (due to either quit or layoff) according to HR records.\footnote{We made a judicious choice of this definition of medium absence. We run a robustness check on our main results using an alternative definition of medium absence--not providing visits in any office for 6 to 20 consecutive days--and find very similar results in Table~\ref{tab:iv_robust} in Appendix~\ref{appendix:lws_dist_alt}.
}\footnote{Since home health visits entail mobility as a nature of work, a provider is not constrained to work for only one office and can visit patients in different offices.
}
The instrument vector $B_{ikt}$ includes the above absence indicator variables (2)-(6) with ``Active'' the omitted category.

For the validity of the instrument set $B_{ikt}$, we require that $Corr(B_{ikt}, [ H_{ikt}, \hat{V}_{ikt} ]' ) \ne 0$ and $B_{ikt} \perp U_{ikt}.$
For the former, it is intuitive that handoffs and missing a visit will be more likely to occur on day $t$ if the nurse who visited the patient in the last visit is unavailable to visit her again on that day.
Table~\ref{tab:lws_dist} presents the distribution of the number of patient episode-day observations as well as the likelihood of having a nurse handoff, a nurse visit, and a hospital readmission for each availability category.\footnote{We provide the same table using an alternative definition of absence categories in Table~\ref{tab:lws_dist_alt} in Appendix~\ref{appendix:lws_dist_alt}.
}
In 65\% of patient-day observations, the last nurse who visited a patient is available for the same office. As expected, the probability of handoff is lowest at 21\% in this subsample compared to all other states corresponding to providers' unavailability.
Handoffs occur in 60-70\% of patient-day observations in which the nurse who visited the patient in a preceding visit is either on medium or long absence.
Similarly, when the last nurse who visited the patient in a preceding visit is unavailable, the patient is less likely to have a nurse visit.
The probability of having a nurse visit is approximately 10\% or smaller when the last nurse is unavailable while the same probability is 27\% when the last nurse is around working at the same office.
This comparison suggests a strong correlation between nurses' unavailability and handoffs.
Consistent with this, we find first-stage results to be quite strong, as shown later in Table~\ref{tab:iv}.

As for the exclusion restriction, we rely on the notion that nurse inactivity is uncorrelated with unobserved daily patient health conditions or any other unobserved nurse-day level or office-day level characteristics. Such correlation would imply that absence by nurses is linked with their patients' likelihood of readmission.
To assess these possibilities, we discuss several ways in which the exclusion restriction condition could be violated and show evidence alleviating those concerns.

First, a potential concern may be that nurses are more likely to become inactive when their patients' health changes. In particular, nurses with patients who are getting progressively sicker may experience burnout and desire a day off. To assess such potential scenarios, Figure~\ref{fig:severity_byday} plots three key measures of patients' severity by number of days prior to nurse inactivity for the stock of patients under the nurse's supervision.
These measures are Charlson comorbidity index, overall status likely to remain fragile, and taking 5 or more medications, as reported in the initial OASIS assessment conducted for each patient.\footnote{We find these measures to be top predictors of rehospitalization. We report the correlations between each of these variables and the indicator for rehospitalization in Table~\ref{tab:ols_corr_severity} in Appendix~\ref{appendix:corr_severity}.}
For this exercise, we define each nurse's set of patients on each day as those who are currently or last visited by the nurse and who are not handed off to another nurse, rehospitalized, or discharged on that day. We separately report these measures against the number of days prior to inactivity by whether the inactivity is short absence (i.e. not providing any visits for 1-2 consecutive days), medium absence (i.e. not providing any visits for 3-14 consecutive days), and long absence/attrition (i.e. not providing any visits for 15 or more consecutive days or exiting the workforce).
The variation in severity measures is driven by compositional changes in types of patients under a nurse's care.
Figure~\ref{fig:severity_byday} shows that there is little variation in patients' severity leading up to nurse inactivity, and if anything, most plots show a slight decline in measures in the day or two leading to absence, suggesting that nurses are more likely to take days off when their patient base is stable.
Alternatively, it could be the case that offices assign less severe cases to nurses expecting to become inactive. Patients in better health are less likely to be readmitted to the hospital, and if inactivity-driven handoffs are more likely to occur for healthier patients, the effect of handoff on readmission is likely to be biased toward zero. Thus, our estimate of the effect would be conservative.
Furthermore, these trends indicate that as nurses approach a period of inactivity, they do not selectively discharge healthier patients and subsequently raise the severity mix of the remaining patients under their care.

Second, there is a concern than nurses' burnout from working extended hours and seeing many patients per day may induce them to become inactive, and at the same time, fatigue and burnout may adversely affect patient outcomes \citep{Aiken2002}. Figure~\ref{fig:dnv_byday2abs} plots the number of patient visits per day as a function of days prior to inactivity. We find that nurses' workload consistently declines before a period of inactivity, with 6 to 7 patient visits per day more than a week prior to inactivity and less than 3 patient visits per day in the three days leading to absence.\footnote{The reduction in workload before a period of inactivity may be a sign of higher levels of nurses' burnout, through missed appointments. In this case, burnout may reinforce the positive association between handoffs and readmission. Nevertheless, the bias may work in the opposite direction. Burnout is often viewed as a process and is likely to manifest itself in quality deteriorations that may result in hospital readmission while nurses are \textit{active}. Then a handoff may be better for patients than being seen by an active burned out nurse. Moreover, the harmful effect of burnout may be realized even before handoffs occur, in which case those patients would drop out of the sample before we can attribute their readmission to handoffs.}




Finally, nurses might be more likely to be absent during high-workload days while high workload, combined with more absences, could result in higher readmissions if quality of care was deteriorated.
For example, \citet{Green2013} found that hospital nurses anticipated high-workload days and strategically elected to take time off from work on those days.\footnote{Workload in a hospital is likely unrelated to nurse staffing, that is, patients are not turned down by hospitals due to temporary fluctuation in nurse staffing. On the other hand, absence in home health is likely to affect the agency's ability to take on new patients and meet the care plan for existing patients. This suggests that the anticipated caseload may be systematically higher than the realized one.}
However, we find no evidence of increases in office-level daily caseload, arrival of new patients, or number of nurse visits immediately following the onset of absence.
Figure~\ref{fig:officeload_byday} plots three office-day level measures: the total number of active patients (i.e. the stock of ongoing episodes), the number of new (admitted) patients, and the total number of nurse visits. These measures are plotted in the 10 days leading to a nurse absence and the 10 days following the onset of a nurse absence.
We find no evidence of an increase in the office stock of episodes prior to absence.

Not surprisingly, absence reduces offices' capacity for taking on new cases by about 30\% in the first two days following the onset of absence, although by the third day office are back to pre-absence levels. Similarly, the total number of nurse visits falls following absence. These findings strengthen the case for our instrument. Absence does not affect the stock of episodes, but reduces the number of visits, hence fewer patients are visited due to absence and those who are visited are likely to experience a handoff. Contrary to \citet{Green2013} we find that nurses are inactive when office-level workload is stable, number of daily visits is lower and fewer patients discharged from hospitals are seen for the first time by the office. In summary, we expect nurse absence in a given day to be positively correlated with the likelihood of handoffs and negatively correlated with the likelihood of visits, but presumably uncorrelated with other factors influencing the likelihood of hospital readmissions. To the extent that different lengths and types of absence provide an exogenous source of variation in the likelihood of handoffs, changes in hospital readmissions should not be driven by nurse absences.


\section{Results on the Effects of Handoffs on the Likelihood of Rehospitalization}
\label{sec:results}

\subsection{Baseline Results from the Cross-Sectional Estimation} \label{sec:ols}


Table~\ref{tab:ols_handoffonly} shows the baseline coefficient estimates on the handoff state indicator from our cross-sectional analysis using four specifications representing different degrees of model saturation, incrementally introducing additional patient level controls.
In all columns, we control for the indicator for having a nurse visit $V_{ikt}$ and variables in the patient-day level vector $X_{ikt}$, office-day level vector $W_{kt}$, day level time fixed effects vector $D_t$, and office fixed effects $\theta_k$, as described in Section~\ref{sec:spec}.
In Columns (2)-(4), we incrementally control for the hospitalization risk, demographic, and comorbidity factors, respectively, whose detailed description is provided above.
In all these columns, the reference category is the case of experiencing no handoffs. All reported standard errors allow for arbitrary correlation among patient-day observations within the same office.

We find that within home health day, patients experiencing nurse handoffs are 0.17 percentage points or 24\% more likely to have hospital readmissions in all specifications.
When restricting to 30-day rehospitalizations, the basis for hospital readmission penalties, the effects of nurse handoffs are slightly lower at 21\% increase (see Appendix~\ref{appendix:hosp30}). For robustness, we also estimate the effects using a fixed effect conditional logit estimation to account for the binary nature of our dependent variable. These results are reported in Appendix~\ref{appendix:clogit}.


\subsection{Results from the Instrumental Variables (IV) estimation} \label{sec:iv}

Table~\ref{tab:iv} reports IV estimation results from using the vector of instruments we discuss in Section~\ref{sec:availability}.\footnote{Table~\ref{tab:iv_robust} in Appendix~\ref{appendix:lws_dist_alt} shows the IV estimation results using an alternative definition of medium absence, as explained in Section~\ref{sec:availability}.
}
We estimate a two-stage least square (2SLS) using a two-step efficient generalized method of moments (GMM) estimator. Panels A and B report the first-stage results for the indicator variables for having a nurse handoff and for a nurse visit, respectively.
We find the vector of instrumental variables to be a strong predictor of both endogenous variables regardless of specification.
The first-stage F-statistic values are large (above $537$ for handoffs and above $213$ for visits ).
In Panel A, not surprisingly, each one of the instruments has a statistically significant positive association with handoffs, with longer absence periods (medium, long and permanent absence) resulting in a greater likelihood of handoffs.
Similarly, in Panel B, each instrumental variable is negatively associated with the probability of the patient receiving a nurse visit, although the difference in the magnitude of the coefficient estimates is smaller.
Panel C reports the second-stage results regressing hospital readmission per-patient-episode-day on the predicted likelihood of handoff and the predicted likelihood of a visit.
We find handoffs to raise hospital readmissions, with a statistically significantly coefficient of that is more than twice in magnitude compared to our cross-sectional findings--54\% versus 24\%.
There is no statistically significant residual effect of having a skilled nursing visit on the probability of hospital readmission.
Panel C reports the p-values for the Sargan-Hansen J-statistic values of 0.4-0.5, which suggests that we cannot reject the null hypothesis that the instrument set is exogenous.




\section{Mechanisms} \label{sec:mechanism}

To explore potential mechanisms behind the overall effect of handoffs on the likelihood of hospital readmissions we find in Section~\ref{sec:results}, we explore three potential mechanisms. First, we examine whether handoffs affect hospital readmissions differently by underlying patient severity. Second, we decompose the effect by the frequency and sequencing of handoffs. Third, we focus on a subset of home health visits which resulted in a hospital readmission and examine whether handoffs affect the number of days from last home health visit to readmission.

\subsection{The Heterogeneous Effect of Handoffs by Patient Severity}
To test for heterogeneous effects of handoffs, we re-estimate the main model~\ref{eq:regeq} for three risk groups of patients. The first group includes all patients indicated as temporarily facing high risks. The second group includes all patients indicated as likely to remain in fragile condition. The third group includes all patients taking five or more medications. The first two groups are mutually exclusive with the first group showing relatively lower level of severity than the second. Table~\ref{tab:iv_byseverity} presents the results from the most saturated specification for the three groups above.

Results for the first group are very similar to the original results reported for all patients in Table~\ref{tab:iv}. In comparison, the second group of fragile patients in column 2 and the third group of patients who were reliant on five medications or more---typically due to multiple chronic conditions---in column 3, were more adversely affected by handoffs.
Fragile patients and patients taking multiple medications were 59\% and 57\% more likely to be readmitted after experiencing handoffs, respectively.


\subsection{Decomposition of the Effect of Handoffs by Frequency and Sequencing}
Patients who experience more handoffs within the same time window may be more likely to become at risk of rehospitalization because the potential harm from provider switches is magnified.
Nevertheless, the sequence of handoffs may matter.
For example, the first handoff may have a weak effect on rehospitalization but when a critical mass of handoffs takes place, the patient's risk of rehospitalization could increase, suggesting a convex relationship between the handoff number and readmissions.
Conversely, it may be the case that the first handoff is the most important one, suggesting a concave relationship between the handoff number and the risk of readmissions. This could happen, for example, if offices are more likely to provide discontinuous care to relatively healthier patients.


In Table~\ref{tab:ols_mechanism}, we report the coefficient estimates of interaction terms between the handoff indicator and four frequency indicators for the first, second, third, or fourth and beyond handoffs.
Since we include home health day fixed-effects, we are separately comparing patients who experienced their first, second, third or fourth handoff states to those who were not in a handoff state (omitted category) on the same home health day.
We find that experiencing one to three handoffs all have a statistically significant effect on the likelihood of readmission. The first handoff is associated with a 35\% increase in hospital readmission, the second handoff a 18\% increase and the third a 16\% increase.
The effect of experiencing a fourth handoff and beyond is not statistically significantly different from patients experiencing no handoffs.
These results suggest a concave relationship between the frequency of handoffs and the likelihood of readmission, with a decreasing adverse marginal effect of handoffs. It appears that healthier patients tend to get less continuous care and experience more handoffs.
Note that this finding is not mechanically driven by sicker patients receiving more visits and having more handoff chances on the same home health day since we control for the cumulative number of nurse visits on each day.

\subsection{The Effect of Handoffs on the Number of Days to Readmission}

Next we investigate the effect of handoffs on time to hospital readmission from the last visit among all patients who had a readmission. We restrict our sample to 7,269 visits that were followed by a hospital readmission. We divide these visits into those involving a handoff and those who did not. Figure~\ref{fig:days2readmit} presents a quantile-quantile plot comparing the distributions of the days-to-readmission from the last nurse visit of patients with and without a nurse handoff. As points lie mostly below the 45-degree line, handoffs are associated with shorter days-to-readmission.

In addition, we repeat our specification in equation~\ref{eq:regeq} by replacing the readmission indicator with days-to-readmission.
For patient-day or office-day level variables, we use data for the day of last nurse visit. Table~\ref{tab:days2readmit_iv} reports the results, indicating 9\% reduction in the number of days from last visit to hospital readmission following a handoff. This decrease is small in absolute magnitude given the mean number of days to readmission of around 4 days. However, our estimated effects are statistically significant at the 1 percent level.







\section{Discussions and Conclusions} \label{sec:conclusion}

Greater continuity and coordination of care are an important mechanism in preventing hospital readmissions \citep{Naylor1999}, and post-acute care providers can achieve it through workforce assignment strategies prioritizing care continuity. However, there is little research on the role of workforce assignment affecting care continuity post-discharge. This paper takes a first step in filling this gap by examining the plausibly causal effect of discontinuity of post-acute care caused by provider switches on hospital readmissions using a novel data set from a large multi-state freestanding home health agency.


Our findings highlight the importance of care continuity prioritization through worker assignment in improving a key competitive performance metric desirable to many health care systems---reduction in hospital readmissions. Studying the elderly Medicare population, we find handoffs to increase hospital readmissions by 54\% when instrumenting for handoffs using breaks in availability of previously assigned nurses.
This estimate implies that a single handoff in home health increases the patients' likelihood of 30-day readmission by 16 percent while the 30-day hospital readmission rate has become a key marker of health care quality and performance as recent payment reforms tie the Medicare payment to hospitals with it.
 Moreover, a calibration exercise suggests that one in four hospital readmissions during a home health episode would be avoided if nurse handoffs were completely eliminated.
Furthermore, in our analysis of potential mechanisms, we find that handoffs are more detrimental for more severe patients. We also find that patients experiencing handoffs for the first time are more likely to have a hospital readmission relative to those experiencing such handoffs for the second and third time. Finally, among patients readmitted to the hospital, those experiencing a handoff on their last visit prior hospitalization were found to have a readmission faster.
These findings suggest that preventing handoffs altogether would be an effective way to reduce hospital readmissions and improve quality of care, and have important implications for scheduling strategies, contracting priorities and regulatory oversight.

Our IV estimates of the handoffs effects are identified from different lengths and types of nurse inactivity. To have a richer understanding of the implication of our results, it would be useful to examine what types of handoffs are affecting hospital readmissions the most. In particular, distinguishing, for example, unexpected/unplanned handoffs that are accompanied by poor transmission of information between nurses from other handoffs may be important, as the former may be more detrimental to patient health outcomes. However, due to data limitations, we cannot distinguish between planned and unplanned inactivity and similarly we cannot distinguish between planned and unplanned handoffs. While one may hypothesize that nurse inactivity is likely to result in unplanned handoffs, this is not at all obvious. First, planned inactivity is likely to involve a planned handoff. Moreover, unplanned handoffs, potentially triggered by worsening of a patient's health status, can occur when the previously assigned nurse is active but busy seeing other patients or when the nurse has a planned absence. Nevertheless, this paper shows that handoffs, whether planned or unplanned, generate discontinuity of care and increase hospital readmissions.

Given the negative impact of handoffs on quality of care, it is important to understand how costly preventing handoffs is. First, it is important to note again that, in theory, handoffs can be eliminated in non-24/7 coverage situations. Nurses visit patients every couple of days, and in theory, there is no reason why the same nurse-patient pair would not remain constant throughout an episode of care.

Of course, in practice, a number of elements makes this situation costly to implement. First, nurse availability maybe disrupted due to planned and unplanned leave, reassignment to other offices, and job separation. Second, operational efficiency and reimbursement policy calls for assigning nurses to tight geographical clusters, designed to minimize time spent traveling between patients. This means that the arrival of new patients may lead to re-optimization of distance algorithms and result in reshufflings of patients among nurses (i.e. handoffs). Lastly, poor patient-nurse match or unplanned patients' health needs may require home health offices to introduce new nurses into the care of patients, again, resulting in handoffs. These drivers of handoffs suggest that avoiding handoffs may be costly and in some case undesirable. Put differently, even though, on net, handoffs result in readmissions, some handoffs may benefit the patient. The solutions may involve lowering the nurses' caseload to allow more flexibility in scheduling, however, this would require hiring more nurses. Other solutions may involve the use of virtual health solutions, remote monitoring, telehealth technology, as well as a mix of skilled and unskilled workers may mitigate discontinuities in care caused by handoffs. Of course, these solutions are costly, their effectiveness has not been broadly tested and the vast majority of insurers would not reimburse home health providers for their provision. As healthcare shifts from volume-based contracts to value based contracts, tying home health payment to performance measures such as reductions in hospital readmissions has the potential of leading home health companies to incorporate some of the solutions above.






\newpage

\bibliography{\myreferences}{}
\bibliographystyle{te}


\begin{singlespace}

\graphicspath{ {/Users/kimk13/Dropbox/Wharton/Research/Labor/gph/anhandoff/} }
\newpage
\begin{figure}[H]
\centering
\includegraphics[width=0.6\linewidth]{nrehosp_epi.pdf}
\begin{minipage}{\linewidth}
% \vskip\baselineskip
\footnotesize
\justify
\emph{Notes:}
Most first skilled nurse visits and rarely second skilled nurse visits occur within the first 5 days, leading to most patients experiencing no handoffs and dropping out from the sample in this region. Thus, we exclude the first five days of home health care in this plot.
\end{minipage}
\caption{The Number of Ongoing Episodes and Readmissions by Home Health Day}
\label{fig:pct_rehosp}
\end{figure}

\graphicspath{ {/Users/kimk13/Dropbox/Wharton/Research/Labor/gph/anhandoff/} }
\begin{figure}[H]
\begin{minipage}{\linewidth}
\centering
\includegraphics[width=0.7\linewidth]{fr_cho_gt2.png}
\footnotesize
\justify
\end{minipage}
\caption{Fraction of Patient Episodes with at Least One, Two, Three, or Four Handoffs by Home Health Day}
\label{fig:fr_cho_gt}
\end{figure}

\graphicspath{ {/Users/kimk13/Dropbox/Wharton/Research/Labor/gph/anhandoff/} }
\begin{figure}[H]
\centering
\includegraphics[width=0.7\linewidth]{frv_ho2.png}
\begin{minipage}{\linewidth}
\footnotesize
\justify
\emph{Notes:}
Most first skilled nurse visits and rarely second skilled nurse visits occur within the first 5 days, leading to most patients experiencing no handoffs and dropping out from the sample in this region. Thus, we exclude the first five days of home health care in this plot.
\end{minipage}
\caption{Fraction of Patient-Days with Nurse Handoffs Conditional on Having a Nurse Visit}
\label{fig:frv_ho}
\end{figure}



\graphicspath{ {/Users/kimk13/Dropbox/Wharton/Research/Labor/gph/anhandoff/} }
\begin{figure}[H]
\centering
\subcaptionbox{Charlson comorbidity score\label{fig:charls_byday2abs}}{\includegraphics[width=0.5\linewidth, keepaspectratio]{charls_byday2abs}}
\subcaptionbox{Overall status\label{fig:overallst_byday2abs}}{\includegraphics[width=0.5\linewidth, keepaspectratio]{overallst_byday2abs}}
\subcaptionbox{Taking multiple medications\label{fig:med_byday2abs}}{\includegraphics[width=0.5\linewidth, keepaspectratio]{med_byday2abs}}
\begin{minipage}{\linewidth}
\footnotesize
\justify
\emph{Notes:} Mean values are plotted.
The inactivity categories are defined as follows:
(1) Short absence---not providing visits in any office for 1 to 2 consecutive days;
(2) Medium absence---not providing visits in any office for 3 to 14 consecutive days;
(3) Long absence/Attrition---not providing visits in any office for 15 or more consecutive days, or not providing visits in any office for 91 or more consecutive days or exiting the workforce (due to either quit or layoff) according to HR records.
The sample used for this figure includes all patients who receive home health care.
\end{minipage}
\caption[Measures of Patient Severity Preceding the Nurse's Inactivity]%
{Measures of Patient Severity Preceding the Nurse's Inactivity}
\label{fig:severity_byday}
\end{figure}

\newpage
\begin{figure}[H]
\centering
\includegraphics[width=0.7\linewidth]{dnv_byday2abs}
\begin{minipage}{\linewidth}
\footnotesize
\justify
\emph{Notes:} Mean values are plotted.
The inactivity refers to any lengths of absence (i.e. not providing visits temporarily) and attrition (i.e. not providing visits permanently).
The sample used for this figure includes all patients who receive home health care.
\end{minipage}
\caption{Daily Workload of Nurses Preceding the Nurse's Inactivity}
\label{fig:dnv_byday2abs}
\end{figure}






\graphicspath{ {/Users/kimk13/Dropbox/Wharton/Research/Labor/gph/anhandoff/} }
\newpage
\begin{figure}[H]
\centering
\subcaptionbox{Daily number of ongoing episodes\label{fig:allepi_byday2abs}}{\includegraphics[height=0.33\linewidth, keepaspectratio]{allepi_byday2abs}}
\vskip\baselineskip
\subcaptionbox{Daily number of new episodes\label{fig:newepi_byday2abs}}{\includegraphics[height=0.33\linewidth, keepaspectratio]{newepi_byday2abs}}
\vskip\baselineskip
\subcaptionbox{Daily number of nurse visits\label{fig:nv_SN_byday2abs}}{\includegraphics[height=0.33\linewidth, keepaspectratio]{nv_SN_byday2abs}}
\begin{minipage}{\linewidth}
\footnotesize
\justify
\emph{Notes:} Mean values are plotted. The inactivity refers to any lengths of absence (i.e. not providing visits temporarily).
The sample used for this figure includes all patients who receive home health care.
\end{minipage}
\caption[Daily Office Caseload Before and After the Nurse's Inactivity]%
{Daily Office Caseload Before and After the Nurse's Inactivity}
\label{fig:officeload_byday}
\end{figure}


\graphicspath{ {/Users/kimk13/Dropbox/Wharton/Research/Labor/gph/anhandoff/} }
\newpage
\begin{figure}[H]
\begin{minipage}{\linewidth}
\centering
\includegraphics[width=0.7\linewidth]{days2readmit3.pdf}
\footnotesize
\justify
\end{minipage}
\caption{Number of days to readmission from the last nurse visit}
\label{fig:days2readmit}
\end{figure}



\newpage
{\footnotesize
\begin{longtable}{lcc}
\caption{Summary Statistics for the Sample Period 2012--2015}
\label{tab:summstats}\\
\toprule
Variables & Mean & Std Dev \\
\midrule
\multicolumn{3}{l}{A. Office-day-level variables (Number of observations = 92,676)} \\
Number of ongoing episodes  &  113.098 & 68.345 \\
Number of active nurses      & 14.596   & 10.908 \\
\\
\multicolumn{3}{l}{B. Patient episode-level variables (Number of observations = 43,740)} \\
Hospital readmission & 0.166 & 0.372 \\
Hospital readmission within 30 days of hospital discharge & 0.130 & 0.337 \\
Death & 0.003 & 0.053 \\
Length of episode (in days) & 32.672 & 16.268 \\
Number of nurse visits & 5.791 & 3.067 \\
Number of nurse handoffs & 1.327 & 1.600 \\
Mean number of days between nurse visits & 5.251 & 2.948 \\
Age & 78.961 & 8.423 \\
Female & 0.598 & 0.490 \\
White & 0.820 & 0.384 \\
Living alone & 0.234 & 0.423 \\
No assistance available & 0.017 & 0.131 \\
Enrolled in per-visit paying Medicare Advantage & 0.190 & 0.393 \\
Enrolled in per-episode paying Medicare Advantage & 0.062 & 0.242 \\
Dual eligible & 0.006 & 0.078 \\
Risk for hospitalization: History of 2+ falls & 0.255 & 0.436 \\
Risk for hospitalization: 2+ hospitalizations & 0.372 & 0.483 \\
Risk for hospitalization: Recent decline in Mental & 0.068 & 0.251 \\
Risk for hospitalization: Take 5+ medications & 0.872 & 0.334 \\
Risk for hospitalization: Other & 0.091 & 0.288 \\
Acute myocardial infarction (AMI) & 0.022 & 0.148 \\
Congestive heart failure (CHF) & 0.130 & 0.336 \\
Peripheral vascular disease (PVD) & 0.016 & 0.125 \\
Cerebrovascular disease (CEVD) & 0.051 & 0.220 \\
Dementia & 0.007 & 0.084 \\
Chronic pulmonary disease (COPD) & 0.104 & 0.305 \\
Rheumatic disease & 0.001 & 0.030 \\
Peptic ulcer disease & 0.003 & 0.055 \\
Mild liver disease & 0.004 & 0.065 \\
Diabetes & 0.017 & 0.129 \\
Diabetes + Complications & 0.009 & 0.096 \\
Hemiplegia or paraplegia (HP/PAPL) & 0.002 & 0.048 \\
Renal disease & 0.029 & 0.169 \\
Cancer & 0.070 & 0.255 \\
Moderate/severe liver disease & 0.002 & 0.045 \\
Metastatic cancer & 0.008 & 0.089 \\
AIDS/HIV & 0.000 & 0.011 \\
Overall status: (Very bad) Progressive conditions & 0.033 & 0.179 \\
Overall status: (Bad) Remain in fragile health & 0.274 & 0.446 \\
Overall status: Temporarily facing high health risks & 0.615 & 0.487 \\
High risk factor: Alcohol dependency & 0.024 & 0.154 \\
High risk factor: Drug dependency & 0.007 & 0.083 \\
High risk factor: Heavy smoking & 0.133 & 0.340 \\
High risk factor: Obesity & 0.163 & 0.369 \\
\bottomrule
\pagebreak
\multicolumn{3}{c}%
{\tablename\ \thetable\ -- \textit{Continued}} \\
\toprule
& Mean & Std Dev \\
\midrule
Pre-HHC condition: Disruptive behavior & 0.010 & 0.102 \\
Pre-HHC condition: Impaired decision-making & 0.149 & 0.356 \\
Pre-HHC condition: Indwelling/Suprapublic catheter & 0.018 & 0.134 \\
Pre-HHC condition: Intractable pain & 0.113 & 0.317 \\
Pre-HHC condition: Memory loss & 0.104 & 0.306 \\
Pre-HHC condition: Urinary incontinence & 0.305 & 0.460 \\
\\
\multicolumn{3}{l}{C. Patient episode-day-level variables (Number of observations = 1,031,904)} \\
Hospital readmission & 0.007 & 0.084 \\
Hospital readmission within 30 days of hospital discharge & 0.006 & 0.074 \\
Handoff & 0.265 & 0.441 \\
First handoff & 0.116 & 0.320 \\
Second handoff & 0.073 & 0.260 \\
Third handoff & 0.038 & 0.190 \\
Fourth+ handoff & 0.038 & 0.192 \\
Handoff from salaried to salaried & 0.084 & 0.278 \\
Handoff from salaried to piece-rate & 0.024 & 0.152 \\
Handoff from piece-rate to piece-rate & 0.006 & 0.074 \\
Handoff from piece-rate to salaried & 0.021 & 0.142 \\
Have a nurse visit & 0.203 & 0.402 \\
Number of days since last nurse visit & 4.957 & 5.232 \\
Number of days since last visit by any provider & 2.742 & 2.771 \\
Cumulative number of nurse visits provided & 4.758 & 2.629 \\
Number of times the current/latest nurse has previously seen the patient & 2.544 & 2.400 \\
Cumulative number of unique nurses the patient has seen & 1.863 & 0.911 \\
Home health day & 20.479 & 12.849 \\
\bottomrule
\end{longtable}
}





\clearpage
\begin{table}[H]
\centering
\footnotesize
\caption{Distribution of the Number of Unique Nurses in Each Episode}
\label{tab:nsn}
\begin{threeparttable}
\begin{tabular}{ccc}
\toprule
Number of unique nurses & Number of episodes & Percent \\
\midrule
1 & 16,705 & 38.19 \\
2 & 16,918 & 38.68 \\
3 & 7,150  & 16.35 \\
4 & 2,148  & 4.91  \\
5 & 603    & 1.38  \\
6 & 155    & 0.35  \\
7 & 40     & 0.09  \\
8 & 18     & 0.04  \\
9 & 3      & 0.01 \\
\midrule
                        & 43,740             & 100.00 \\
\bottomrule
\end{tabular}
	\begin{tablenotes}[para,flushleft]
	\footnotesize
	\item \emph{Notes.} The sample excludes episodes with only 1 nurse visit or more than 20 nurse visits provided, and episodes with more than 15 nurse handoffs.
	\end{tablenotes}
\end{threeparttable}
\end{table}


\clearpage
\begin{table}[H]
\centering
\footnotesize
\caption{Distribution of Patient-Day Observations and the Likelihood of Nurse Handoff, Nurse Visit, and Readmission by the Availability of Nurse Who Visited a Patient in the Last Visit}
\label{tab:lws_dist}
\begin{threeparttable}
\begin{tabular}{lccccc}
\toprule
& N Obs                    & \% Obs & \% Handoff & \% Have a nurse visit & \% Readmission   \\
\midrule
Active                   & 670,621 & 64.99      & 20.63            & 27.37          & 0.77 \\
Short absence (1-6 days) & 290,098 & 28.11      & 31.07            & 6.20           & 0.56 \\
Medium absence (7-14 days)     & 32,232  & 3.12       & 62.45            & 11.94          & 0.67 \\
Long absence (15+ days)  & 13,834  & 1.34       & 73.28            & 11.41          & 0.71  \\
Assigned to other office     & 13,415  & 1.30       & 49.30            & 11.20          & 0.83 \\
Attrition                & 11,704  & 1.13       & 65.02            & 9.28           & 0.67 \\
\midrule
Total                    & 1,031,904 & 100.00     &                &                  &     \\
\bottomrule
\end{tabular}
	\begin{tablenotes}[para,flushleft]
	\item \emph{Notes.} In the entire sample of patient-day observations, the percentage of handoff is 26.46\%; the percentage of having a nurse visit is 20.31\%; the percentage of readmission is 0.71\%.
	\end{tablenotes}
\end{threeparttable}
\end{table}



\clearpage
\begin{table}[H]
\footnotesize
\setlength\tabcolsep{0pt}
\centering
\caption{The Effect of Handoffs on the Likelihood of Rehospitalization}
\label{tab:ols_handoffonly}
\begin{threeparttable}
\begin{tabular*}{\textwidth}{l@{\extracolsep{\fill}}*{4}{c}}
\toprule
& \multicolumn{4}{c}{Dep Var: Indicator for being rehospitalized} \\
 & (1) & (2) & (3) & (4) \\
\midrule
Handoff & 0.0017*** & 0.0017*** & 0.0017*** & 0.0017*** \\
 & (0.0002) & (0.0002) & (0.0002) & (0.0002) \\
R-squared & 0.0067 & 0.0072 & 0.0073 & 0.0081 \\
Observations & 1,031,904 & 1,031,904 & 1,031,904 & 1,031,904 \\
Hospitalization risk controls & . & Yes & Yes & Yes \\
Demographic controls & . & . & Yes & Yes \\
Comorbidity controls & . & . & . & Yes \\
\bottomrule
\end{tabular*}
	\begin{tablenotes}[para,flushleft]
	\scriptsize
	\item \emph{Notes.} This table presents OLS estimates of the effect of experiencing a handoff on the likelihood of rehospitalization.

	An observation is a patient episode-day. Standard errors are clustered at the home health office level, allowing for arbitrary correlation among episode-days in the same office in parentheses.
	In all panels and all columns, we control for the indicator for having a nurse visit, cumulative number of nurse visits provided, mean interval of days between two consecutive visits during the episode, number of days since last nurse visit, number of days since last visit by any provider; number of ongoing episodes in the office-day, number of nurses working in the office-day; and office fixed effects, day of week fixed effects, month-year of the day fixed effects, and home health day fixed effects.
	In columns (2)-(4), hospitalization risk controls include dummies for the risk factors for hospitalization: history of 2+ falls in the past 12 months; 2+ hospitalizations in the past 6 months; decline in mental, emotional, or behavioral status in the past 3 months, currently taking 5+ medications, and others.
	In columns (3)-(4), demographic controls include age dummies for each age 66-94 and age 95+ (reference group: age=65); female; white; insurance type--Medicare Advantage (MA) plan with a visit-based reimbursement, MA plan with an episode-based reimbursement, dual eligible with Medicaid enrollment (reference group: Medicare FFS); dummy for having no assistance available; dummy for living alone.
	In column (4), comorbidity controls include dummies for 17 Charlson comorbidity index factors; dummies overall status--very bad (patient has serious progressive conditions that could lead to death within a year), bad (patient is likely to remain in fragile health); and temporarily bad (temporary facing high health risks); dummies for high-risk factors--alcohol dependency, drug dependency, smoking, obesity--and dummies for conditions prior to hospital stay within past 14 days--disruptive or socially inappropriate behavior, impaired decision making, indwelling or suprapublic catheter, intractable pain, serious memory loss, urinary incontinence.
	 *significant at 10\%; **significant at 5\%; ***significant at 1\%.
	\end{tablenotes}
\end{threeparttable}
\end{table}




\begin{table}[H]
\footnotesize
\setlength\tabcolsep{0pt}
\centering
\caption{The Effect of Handoffs on the Likelihood of Rehospitalization When Using Instruments}
\label{tab:iv}
\begin{threeparttable}
{
\def\sym#1{\ifmmode^{#1}\else\(^{#1}\)\fi}
\begin{tabular*}{\textwidth}{l@{\extracolsep{\fill}}*{4}{c}}
\toprule
                    &\multicolumn{1}{c}{(1)}&\multicolumn{1}{c}{(2)}&\multicolumn{1}{c}{(3)}&\multicolumn{1}{c}{(4)}\\
\midrule
\multicolumn{5}{l}{A. First stage - Handoff} \\
Short absence       &      0.1327***&      0.1327***&      0.1326***&      0.1326***\\
                    &    (0.0063)   &    (0.0063)   &    (0.0063)   &    (0.0063)   \\
Medium absence      &      0.4117***&      0.4117***&      0.4115***&      0.4112***\\
                    &    (0.0101)   &    (0.0101)   &    (0.0101)   &    (0.0100)   \\
Long absence        &      0.5301***&      0.5303***&      0.5298***&      0.5294***\\
                    &    (0.0121)   &    (0.0121)   &    (0.0122)   &    (0.0121)   \\
Other office        &      0.3003***&      0.3003***&      0.3001***&      0.3001***\\
                    &    (0.0206)   &    (0.0206)   &    (0.0206)   &    (0.0206)   \\
Attrition           &      0.4583***&      0.4585***&      0.4590***&      0.4587***\\
                    &    (0.0156)   &    (0.0157)   &    (0.0156)   &    (0.0157)   \\
R-squared           &       0.174   &       0.174   &       0.174   &       0.175   \\
F-statistic         &     538.100   &     537.186   &     537.086   &     540.415   \\
\\
\multicolumn{5}{l}{B. First stage - Have a nurse visit} \\
Short absence       &     -0.1105***&     -0.1106***&     -0.1106***&     -0.1106***\\
                    &    (0.0047)   &    (0.0047)   &    (0.0047)   &    (0.0047)   \\
Medium absence      &     -0.0956***&     -0.0956***&     -0.0956***&     -0.0956***\\
                    &    (0.0032)   &    (0.0032)   &    (0.0032)   &    (0.0032)   \\
Long absence        &     -0.0934***&     -0.0934***&     -0.0935***&     -0.0934***\\
                    &    (0.0048)   &    (0.0049)   &    (0.0048)   &    (0.0048)   \\
Other office        &     -0.1321***&     -0.1322***&     -0.1320***&     -0.1320***\\
                    &    (0.0071)   &    (0.0072)   &    (0.0072)   &    (0.0072)   \\
Attrition           &     -0.0784***&     -0.0783***&     -0.0783***&     -0.0778***\\
                    &    (0.0041)   &    (0.0041)   &    (0.0042)   &    (0.0042)   \\
R-squared           &       0.244   &       0.244   &       0.245   &       0.245   \\
F-statistic         &     220.406   &     218.998   &     215.463   &     213.609   \\
\\
\multicolumn{5}{l}{C. Second stage - Rehospitalization} \\
Handoff             &      0.0036***&      0.0037***&      0.0039***&      0.0038***\\
                    &    (0.0011)   &    (0.0011)   &    (0.0011)   &    (0.0011)   \\
Have a nurse visit  &      0.0036   &      0.0037   &      0.0039   &      0.0039   \\
                    &    (0.0027)   &    (0.0027)   &    (0.0027)   &    (0.0026)   \\
R-squared           &       0.004   &       0.005   &       0.005   &       0.006   \\
J-statistic p-value &       0.503   &       0.419   &       0.405   &       0.437   \\
\\
\midrule
Observations        &     1,031,904         &     1,031,904         &     1,031,904         &     1,031,904         \\
Hospitalization risk controls & . & Yes & Yes & Yes \\
Demographic controls & . & . & Yes & Yes \\
 Comorbidity controls & . & . & . & Yes \\
\bottomrule
\end{tabular*}
}
	\begin{tablenotes}[para,flushleft]
	\scriptsize
	\item \emph{Notes.}  This table presents IV estimates of the effect of handoffs on the likelihood of rehospitalization obtained with a two-step efficient generalized method of moments (GMM) estimator.
	We instrument the indicator variables for nurse handoffs and for nurse visits with the following 5 instruments:
(1) Active--visiting patients in the patient's home office;
(2) Short absence--not providing visits in any office for 1 to 6 consecutive days;
(3) Medium absence--not providing visits in any office for 7 to 14 consecutive days;
(4) Long absence--not providing visits in any office for 15 or more consecutive days;
(5) Assigned to other office--providing visits exclusively in a different office; and
(6) Attrition--day post labor termination for nurse according to HR records.
	An observation is a patient episode-day.
	Standard errors are clustered at the home health office level, allowing for arbitrary correlation among episode-days in the same office in parentheses.
		In all specifications, we control for the cumulative number of nurse visits provided, mean interval of days between two consecutive visits, number of days since last nurse visit, number of days since last visit by any provider; number of ongoing episodes in the office-day, number of nurses working in the office-day; and office fixed effects, day of week fixed effects, month-year of the day fixed effects, and home health day fixed effects.
	Columns (2)-(4) additionally control for hospitalization risk controls, demographic controls, and comorbidity controls, respectively.
	*significant at 10\%; **significant at 5\%; ***significant at 1\%.

	\end{tablenotes}
\end{threeparttable}
\end{table}



\begin{table}[H]
\footnotesize
\setlength\tabcolsep{0pt}
\centering
\caption{The Effect of Handoffs on the Likelihood of Rehospitalization among Low-Risk and High-Risk Patients}
\label{tab:iv_byseverity}
\begin{threeparttable}
{
\def\sym#1{\ifmmode^{#1}\else\(^{#1}\)\fi}
\begin{tabular*}{\textwidth}{l@{\extracolsep{\fill}}*{3}{c}}
\toprule
& {Temporarily High Risk} & {Fragile Health} & {Taking 5+ Medications}\\
                    &\multicolumn{1}{c}{(1)}&\multicolumn{1}{c}{(2)}&\multicolumn{1}{c}{(3)}\\
\midrule
\multicolumn{4}{l}{A. First stage - Handoff} \\
Short absence (1-6 days) &      0.1361*** &      0.1248*** &      0.1317***\\
                    &    (0.0066)      &    (0.0078)      &    (0.0062)   \\
Medium absence (7-14 days)&      0.4127*** &      0.4071*** &      0.4110***\\
                     &    (0.0118)       &    (0.0131)    &    (0.0102)   \\
Long absence (15-90 days)&      0.5301*** &      0.5066*** &      0.5294***\\
                   &    (0.0134)      &    (0.0193)     &    (0.0134)   \\
Assigned to other office&      0.3209*** &      0.2625*** &      0.2962***\\
                  &    (0.0217)       &    (0.0265)    &    (0.0209)   \\
Attrition       &      0.4762***  &      0.4433***  &      0.4563***\\
                 &    (0.0190)        &    (0.0282)     &    (0.0153)   \\
R-squared        &       0.178       &       0.179    &       0.175   \\
F-statistic       &     497.345     &     311.131     &     481.122   \\
\\
\multicolumn{4}{l}{B. First stage - Have a nurse visit} \\
Short absence (1-6 days) &     -0.1111*** &     -0.1068*** &     -0.1096***\\
                  &    (0.0050)      &    (0.0054)      &    (0.0046)   \\
Medium absence (7-14 days)&     -0.0963*** &     -0.0966*** &     -0.0956***\\
                  &    (0.0036)     &    (0.0045)     &    (0.0034)   \\
Long absence (15-90 days)&     -0.0945*** &     -0.0891*** &     -0.0920***\\
                 &    (0.0055)      &    (0.0057)      &    (0.0047)   \\
Assigned to other office&     -0.1319*** &     -0.1310*** &     -0.1312***\\
                &    (0.0072)      &    (0.0100)      &    (0.0076)   \\
Attrition        &     -0.0837***  &     -0.0701*** &     -0.0782***\\
                    &    (0.0052)    &    (0.0073)     &    (0.0042)   \\
R-squared           &       0.247   &       0.241    &       0.244   \\
F-statistic       &     163.003        &     173.479     &     199.858   \\
\\
\multicolumn{4}{l}{C. Second stage - Rehospitalization} \\
Handoff         &      0.0034***     &      0.0053*   &      0.0040***\\
                   &    (0.0010)     &    (0.0029)    &    (0.0012)   \\
Have a nurse visit   &      0.0039   &      0.0041    &      0.0048   \\
                    &    (0.0024)     &    (0.0063)    &    (0.0031)   \\
R-squared          &       0.004      &       0.007    &       0.006   \\
J-statistic p-value    &       0.531   &       0.751   &       0.394   \\
\\
\midrule
Observations       &      625,687   &      300,077      &      908,473   \\
Hospitalization risk controls & Yes & Yes & Yes \\
Demographic controls & Yes & Yes & Yes \\
 Comorbidity controls & Yes & Yes & Yes \\
\bottomrule
\end{tabular*}
}
	\begin{tablenotes}[para,flushleft]
	\scriptsize
	\item \emph{Notes.}  This table presents IV estimates of the effect of handoffs on the likelihood of rehospitalization obtained with a two-step efficient generalized method of moments (GMM) estimator.
	We instrument the indicator variables for nurse handoffs and for nurse visits with the following 5 instruments:
(1) Active--visiting patients in the patient's home office;
(2) Short absence--not providing visits in any office for 1 to 6 consecutive days;
(3) Medium absence--not providing visits in any office for 7 to 14 consecutive days;
(4) Long absence--not providing visits in any office for 15 or more consecutive days;
(5) Assigned to other office--providing visits exclusively in a different office; and
(6) Attrition--day post labor termination for nurse according to HR records.
	An observation is a patient episode-day.
	Robust standard errors allowing for arbitrary correlation within the same office in parentheses.
		In all specifications, we control for the cumulative number of nurse visits provided, mean interval of days between two consecutive visits, number of days since last nurse visit, number of days since last visit by any provider; number of ongoing episodes in the office-day, number of nurses working in the office-day; and office fixed effects, day of week fixed effects, month-year of the day fixed effects, and home health day fixed effects.
	Columns (2)-(4) additionally control for hospitalization risk controls, demographic controls, and comorbidity controls, respectively.
	*significant at 10\%; **significant at 5\%; ***significant at 1\%.

	\end{tablenotes}
\end{threeparttable}
\end{table}




\begin{table}[H]
\footnotesize
\setlength\tabcolsep{0pt}
\centering
\caption{The Effect of Handoffs on the Likelihood of Rehospitalization by the Frequency and Sequencing of Handoffs}
\label{tab:ols_mechanism}
\begin{threeparttable}
\begin{tabular*}{\textwidth}{l@{\extracolsep{\fill}}*{4}{c}}
\toprule
& \multicolumn{4}{c}{Dep Var: Indicator for being rehospitalized} \\
 & (1) & (2) & (3) & (4) \\
\midrule
First handoff & 0.0026*** & 0.0026*** & 0.0026*** & 0.0025*** \\
 & (0.0004) & (0.0004) & (0.0004) & (0.0004) \\
Second handoff & 0.0013*** & 0.0013*** & 0.0013*** & 0.0013*** \\
 & (0.0003) & (0.0003) & (0.0003) & (0.0003) \\
Third handoff & 0.0013*** & 0.0012*** & 0.0012*** & 0.0012*** \\
 & (0.0004) & (0.0004) & (0.0004) & (0.0004) \\
Fourth+ handoff & 0.0008 & 0.0008 & 0.0008* & 0.0007 \\
 & (0.0005) & (0.0005) & (0.0005) & (0.0005) \\
R-squared & 0.0067 & 0.0072 & 0.0073 & 0.0082 \\
\midrule
Observations & 1,031,904 & 1,031,904 & 1,031,904 & 1,031,904 \\
Hospitalization risk controls & . & Yes & Yes & Yes \\
Demographic controls & . & . & Yes & Yes \\
 Comorbidity controls & . & . & . & Yes \\
\bottomrule
\end{tabular*}
	\begin{tablenotes}[para,flushleft]
	\scriptsize
	\item \emph{Notes.} This table presents OLS estimates of the effect of the frequency and sequencing of handoffs on the likelihood of rehospitalization.

	An observation is a patient episode-day. Standard errors are clustered at the home health office level, allowing for arbitrary correlation among episode-days in the same office in parentheses.
	In all panels and all columns, we control for the indicator for having a nurse visit, cumulative number of nurse visits provided, mean interval of days between two consecutive visits during the episode, number of days since last nurse visit, number of days since last visit by any provider; number of ongoing episodes in the office-day, number of nurses working in the office-day; and office fixed effects, day of week fixed effects, month-year of the day fixed effects, and home health day fixed effects.
	In columns (2)-(4), hospitalization risk controls include dummies for the risk factors for hospitalization: history of 2+ falls in the past 12 months; 2+ hospitalizations in the past 6 months; decline in mental, emotional, or behavioral status in the past 3 months, currently taking 5+ medications, and others.
	In columns (3)-(4), demographic controls include age dummies for each age 66-94 and age 95+ (reference group: age=65); female; white; insurance type--Medicare Advantage (MA) plan with a visit-based reimbursement, MA plan with an episode-based reimbursement, dual eligible with Medicaid enrollment (reference group: Medicare FFS); dummy for having no assistance available; dummy for living alone.
	In column (4), comorbidity controls include dummies for 17 Charlson comorbidity index factors; dummies overall status--very bad (patient has serious progressive conditions that could lead to death within a year), bad (patient is likely to remain in fragile health); and temporarily bad (temporary facing high health risks); dummies for high-risk factors--alcohol dependency, drug dependency, smoking, obesity--and dummies for conditions prior to hospital stay within past 14 days--disruptive or socially inappropriate behavior, impaired decision making, indwelling or suprapublic catheter, intractable pain, serious memory loss, urinary incontinence.
	 *significant at 10\%; **significant at 5\%; ***significant at 1\%.
	\end{tablenotes}
\end{threeparttable}
\end{table}



\begin{table}[H]
\footnotesize
\setlength\tabcolsep{0pt}
\centering
\caption{The Effect of Handoffs on Time to Rehospitalization from the Last Nurse Visit}
\label{tab:days2readmit_iv}
\begin{threeparttable}
{
\def\sym#1{\ifmmode^{#1}\else\(^{#1}\)\fi}
\begin{tabular*}{\textwidth}{l@{\extracolsep{\fill}}*{4}{c}}
\toprule
                    &\multicolumn{1}{c}{(1)}&\multicolumn{1}{c}{(2)}&\multicolumn{1}{c}{(3)}&\multicolumn{1}{c}{(4)}\\
\midrule
\multicolumn{5}{l}{A. First stage - Handoff} \\
Short absence       &      0.7471***&      0.7476***&      0.7465***&      0.7456***\\
                    &    (0.0161)   &    (0.0162)   &    (0.0163)   &    (0.0163)   \\
Medium absence      &      0.7669***&      0.7659***&      0.7676***&      0.7691***\\
                    &    (0.0222)   &    (0.0223)   &    (0.0228)   &    (0.0223)   \\
Long absence        &      0.7653***&      0.7656***&      0.7660***&      0.7645***\\
                    &    (0.0232)   &    (0.0231)   &    (0.0227)   &    (0.0235)   \\
Other office        &      0.7868***&      0.7871***&      0.7921***&      0.7947***\\
                    &    (0.0296)   &    (0.0298)   &    (0.0294)   &    (0.0294)   \\
Attrition           &      0.8075***&      0.8078***&      0.8075***&      0.8055***\\
                    &    (0.0260)   &    (0.0267)   &    (0.0277)   &    (0.0290)   \\
R-squared           &       0.434   &       0.434   &       0.437   &       0.439   \\
F-statistic         &     441.364   &     433.367   &     433.200   &     434.778   \\
\\
\multicolumn{5}{l}{B. Second stage - Ln(Number of days to readmission since last nurse visit)} \\
Handoff             &     -0.0921***&     -0.0906***&     -0.0863***&     -0.0920***\\
                    &    (0.0269)   &    (0.0269)   &    (0.0268)   &    (0.0276)   \\
\midrule
R-squared           &       0.336   &       0.336   &       0.340   &       0.343   \\
J-statistic p-value &       0.033   &       0.036   &       0.032   &       0.032   \\
\midrule
Observations        &        7269   &        7269   &        7269   &        7269   \\
Hospitalization risk controls & . & Yes & Yes & Yes \\
Demographic controls & . & . & Yes & Yes \\
 Comorbidity controls & . & . & . & Yes \\
\bottomrule
\end{tabular*}
}
	\begin{tablenotes}[para,flushleft]
	\scriptsize
	\item \emph{Notes.} This table presents IV estimates of the effect of handoffs on time to rehospitalization from the last nurse visit obtained with a two-step efficient generalized method of moments (GMM) estimator.
	We instrument the indicator variable for nurse handoffs with the following 5 instruments on the day of last nurse visit:
(1) Active--visiting patients in the patient's home office;
(2) Short absence--not providing visits in any office for 1 to 6 consecutive days;
(3) Medium absence--not providing visits in any office for 7 to 14 consecutive days;
(4) Long absence--not providing visits in any office for 15 or more consecutive days;
(5) Assigned to other office--providing visits exclusively in a different office; and
(6) Attrition--day post labor termination for nurse according to HR records.
	An observation is a patient episode.
	Standard errors are clustered at the home health office level, allowing for arbitrary correlation among episode in the same office in parentheses.
		In all specifications, we control for the cumulative number of nurse visits provided, mean interval of days between two consecutive visits, number of days since last nurse visit, number of days since last visit by any provider; number of ongoing episodes in the office-day, number of nurses working in the office-day; and office fixed effects, day of week fixed effects, month-year of the day fixed effects, and home health day fixed effects.
		The day level data refer to the day of last nurse visit.
	Columns (2)-(4) additionally control for hospitalization risk controls, demographic controls, and comorbidity controls, respectively.
	*significant at 10\%; **significant at 5\%; ***significant at 1\%.

	\end{tablenotes}
\end{threeparttable}
\end{table}




\newpage
\section{Appendix}

\subsection{Analysis Using the 30-Day Hospital Readmission Outcome}
\label{appendix:hosp30}

We report the OLS and IV results estimated using the indicator for rehospitalization within 30 days of hospital discharge in Tables~\ref{tab:ols_handoffonly30} and \ref{tab:iv30}, respectively.
We find that the coefficient estimates remain similar, albeit slightly lower in the OLS estimation result and higher in the IV estimation result.
The similar size in estimates is not surprising since Table~\ref{tab:summstats} shows that most of the hospital readmission occurs within 30 days of hospital discharge.
The IV estimates in Table~\ref{tab:iv30} imply that experiencing a handoff increases the probability of readmission by 0.42--0.47 percentage points (70--78\%).
The higher percentage changes in this result result from a lower 30-day readmission rate (13\%) than the all-time readmission rate (17\%).

\clearpage
\begin{table}[H]
\footnotesize
\setlength\tabcolsep{0pt}
\centering
\caption{The Effect of Handoffs on the Likelihood of 30-Day Rehospitalization}
\label{tab:ols_handoffonly30}
\begin{threeparttable}
\begin{tabular*}{\textwidth}{l@{\extracolsep{\fill}}*{4}{c}}
\toprule
& \multicolumn{4}{c}{Dep Var: Indicator for being rehospitalized} \\
 & (1) & (2) & (3) & (4) \\
\midrule
Handoff & 0.0015*** & 0.0015*** & 0.0015*** & 0.0015*** \\
 & (0.0002) & (0.0002) & (0.0002) & (0.0002) \\
R-squared & 0.0074 & 0.0079 & 0.0080 & 0.0088 \\
Observations & 1,031,904 & 1,031,904 & 1,031,904 & 1,031,904 \\
Hospitalization risk controls & . & Yes & Yes & Yes \\
Demographic controls & . & . & Yes & Yes \\
Comorbidity controls & . & . & . & Yes \\
\bottomrule
\end{tabular*}
	\begin{tablenotes}[para,flushleft]
	\scriptsize
	\item \emph{Notes.} This table presents OLS estimates of the effect of handoffs on the likelihood of 30-day rehospitalization.

	An observation is a patient episode-day. Standard errors are clustered at the home health office level, allowing for arbitrary correlation among episode-days in the same office in parentheses.
	In all panels and all columns, we control for the indicator for having a nurse visit, cumulative number of nurse visits provided, mean interval of days between two consecutive visits during the episode, number of days since last nurse visit, number of days since last visit by any provider; number of ongoing episodes in the office-day, number of nurses working in the office-day; and office fixed effects, day of week fixed effects, month-year of the day fixed effects, and home health day fixed effects.
	In columns (2)-(4), hospitalization risk controls include dummies for the risk factors for hospitalization: history of 2+ falls in the past 12 months; 2+ hospitalizations in the past 6 months; decline in mental, emotional, or behavioral status in the past 3 months, currently taking 5+ medications, and others.
	In columns (3)-(4), demographic controls include age dummies for each age 66-94 and age 95+ (reference group: age=65); female; white; insurance type--Medicare Advantage (MA) plan with a visit-based reimbursement, MA plan with an episode-based reimbursement, dual eligible with Medicaid enrollment (reference group: Medicare FFS); dummy for having no assistance available; dummy for living alone.
	In column (4), comorbidity controls include dummies for 17 Charlson comorbidity index factors; dummies overall status--very bad (patient has serious progressive conditions that could lead to death within a year), bad (patient is likely to remain in fragile health); and temporarily bad (temporary facing high health risks); dummies for high-risk factors--alcohol dependency, drug dependency, smoking, obesity--and dummies for conditions prior to hospital stay within past 14 days--disruptive or socially inappropriate behavior, impaired decision making, indwelling or suprapublic catheter, intractable pain, serious memory loss, urinary incontinence.
	 *significant at 10\%; **significant at 5\%; ***significant at 1\%.
	\end{tablenotes}
\end{threeparttable}
\end{table}

\newpage
\begin{table}[H]
\footnotesize
\setlength\tabcolsep{0pt}
\centering
\caption{The Effect of Handoffs on the Likelihood of 30-Day Rehospitalization When Using Instruments}
\label{tab:iv30}
\begin{threeparttable}
{
\def\sym#1{\ifmmode^{#1}\else\(^{#1}\)\fi}
\begin{tabular*}{\textwidth}{l@{\extracolsep{\fill}}*{4}{c}}
\toprule
                    &\multicolumn{1}{c}{(1)}&\multicolumn{1}{c}{(2)}&\multicolumn{1}{c}{(3)}&\multicolumn{1}{c}{(4)}\\
\midrule
\multicolumn{5}{l}{A. First stage - Handoff} \\
Short absence       &      0.0879***&      0.0879***&      0.0879***&      0.0880***\\
                    &    (0.0047)   &    (0.0047)   &    (0.0047)   &    (0.0047)   \\
Medium absence      &      0.2707***&      0.2707***&      0.2706***&      0.2706***\\
                    &    (0.0076)   &    (0.0076)   &    (0.0075)   &    (0.0075)   \\
Long absence        &      0.3462***&      0.3463***&      0.3462***&      0.3460***\\
                    &    (0.0104)   &    (0.0104)   &    (0.0104)   &    (0.0104)   \\
Other office        &      0.2112***&      0.2113***&      0.2111***&      0.2112***\\
                    &    (0.0143)   &    (0.0143)   &    (0.0144)   &    (0.0143)   \\
Attrition           &      0.2763***&      0.2764***&      0.2767***&      0.2767***\\
                    &    (0.0118)   &    (0.0118)   &    (0.0118)   &    (0.0117)   \\
R-squared           &      0.430   &      0.430   &      0.430   &      0.430   \\
F-statistic         &    389.119   &    389.243   &    393.149   &    395.680   \\
\\
\multicolumn{5}{l}{B. First stage - Have a nurse visit} \\
Short absence       &     -0.1115***&     -0.1116***&     -0.1116***&     -0.1116***\\
                    &    (0.0047)   &    (0.0047)   &    (0.0047)   &    (0.0048)   \\
Medium absence      &     -0.0981***&     -0.0982***&     -0.0981***&     -0.0982***\\
                    &    (0.0033)   &    (0.0033)   &    (0.0033)   &    (0.0033)   \\
Long absence        &     -0.0972***&     -0.0972***&     -0.0973***&     -0.0973***\\
                    &    (0.0050)   &    (0.0050)   &    (0.0050)   &    (0.0050)   \\
Other office        &     -0.1344***&     -0.1345***&     -0.1343***&     -0.1343***\\
                    &    (0.0071)   &    (0.0071)   &    (0.0071)   &    (0.0071)   \\
Attrition           &     -0.0816***&     -0.0814***&     -0.0815***&     -0.0810***\\
                    &    (0.0041)   &    (0.0042)   &    (0.0042)   &    (0.0042)   \\
R-squared           &      0.245   &      0.245   &      0.245   &      0.245   \\
F-statistic         &    222.072   &    220.780   &    216.391  &    214.325   \\
\\
\multicolumn{5}{l}{C. Second stage - 30-day rehospitalization} \\
Handoff             &      0.0042***&      0.0044***&      0.0046***&      0.0047***\\
                    &    (0.0016)   &    (0.0016)   &    (0.0016)   &    (0.0016)   \\
Have a nurse visit  &      0.0044*  &      0.0046*  &      0.0047*  &      0.0047*  \\
                    &    (0.0026)   &    (0.0026)   &    (0.0026)   &    (0.0026)   \\
R-squared           &      0.004   &      0.004   &      0.004   &      0.005   \\
J-statistic p-value &      0.744   &      0.669   &      0.650   &      0.742   \\
\\
\midrule
Observations        &     1,031,904         &     1,031,904         &     1,031,904         &     1,031,904         \\
Hospitalization risk controls & . & Yes & Yes & Yes \\
Demographic controls & . & . & Yes & Yes \\
 Comorbidity controls & . & . & . & Yes \\
\bottomrule
\end{tabular*}
}
	\begin{tablenotes}[para,flushleft]
	\scriptsize
	\item \emph{Notes.} This table presents IV estimates of the effect of handoffs on the likelihood of 30-day rehospitalization obtained with a two-step efficient generalized method of moments (GMM) estimator.
	We instrument the indicator variables for nurse handoffs and for nurse visits with the following 5 instruments:
(1) Active--visiting patients in the patient's home office;
(2) Short absence--not providing visits in any office for 1 to 6 consecutive days;
(3) Medium absence--not providing visits in any office for 7 to 14 consecutive days;
(4) Long absence--not providing visits in any office for 15 or more consecutive days;
(5) Assigned to other office--providing visits exclusively in a different office; and
(6) Attrition--day post labor termination for nurse according to HR records.
	An observation is a patient episode-day.
		Robust standard errors allowing for arbitrary correlation within the same office in parentheses.
		In all specifications, we control for the cumulative number of nurse visits provided, mean interval of days between two consecutive visits, number of days since last nurse visit, number of days since last visit by any provider; number of ongoing episodes in the office-day, number of nurses working in the office-day; and office fixed effects, day of week fixed effects, month-year of the day fixed effects, and home health day fixed effects.
		Columns (2)-(4) additionally control for hospitalization risk controls, demographic controls, and comorbidity controls, respectively.
	*significant at 10\%; **significant at 5\%; ***significant at 1\%.

	\end{tablenotes}
\end{threeparttable}
\end{table}


\clearpage
\subsection{Results from the Conditional Logit Model Estimation}
\label{appendix:clogit}

For robustness check, we also estimate a fixed effect conditional logit model to account for the binary nature of our dependent variable.
Table~\ref{tab:clogit} reports average marginal effects estimated using the fixed effect conditional logit estimation results.
These average marginal effects are even stronger than implied by the OLS and IV estimates.




\begin{table}[H]
\footnotesize
\setlength\tabcolsep{0pt}
\centering
\caption{The Effect of Handoffs on the Likelihood of Rehospitalization}
\label{tab:clogit}
\begin{threeparttable}
\begin{tabular*}{\textwidth}{l@{\extracolsep{\fill}}*{4}{c}}
\toprule
& \multicolumn{4}{c}{Dep Var: Indicator for being rehospitalized} \\
 & (1) & (2) & (3) & (4) \\
\midrule
Handoff & 0.227***  &  0.225*** & 0.226*** & 0.216*** \\
 & (0.0281)   & (0.0280)  & (0.0279) & (0.0282)\\

Log likelihood & -35842.873 & -35580.826 & -35534.201  & -35183.389 \\
Observations & 1,031,904 & 1,031,904 & 1,031,904 & 1,031,904 \\
Hospitalization risk controls & . & Yes & Yes & Yes \\
Demographic controls & . & . & Yes & Yes \\
Comorbidity controls & . & . & . & Yes \\
\bottomrule
\end{tabular*}
	\begin{tablenotes}[para,flushleft]
	\scriptsize
	\item \emph{Notes.} This table presents average marginal effects estimated using the fixed effect conditional logit estimation model.

	An observation is a patient episode-day. Standard errors are clustered at the home health office level, allowing for arbitrary correlation among episode-days in the same office in parentheses.
	In all panels and all columns, we control for the indicator for having a nurse visit, cumulative number of nurse visits provided, mean interval of days between two consecutive visits during the episode, number of days since last nurse visit, number of days since last visit by any provider; number of ongoing episodes in the office-day, number of nurses working in the office-day; and office fixed effects, day of week fixed effects, month-year of the day fixed effects, and home health day fixed effects.
	In columns (2)-(4), hospitalization risk controls include dummies for the risk factors for hospitalization: history of 2+ falls in the past 12 months; 2+ hospitalizations in the past 6 months; decline in mental, emotional, or behavioral status in the past 3 months, currently taking 5+ medications, and others.
	In columns (3)-(4), demographic controls include age dummies for each age 66-94 and age 95+ (reference group: age=65); female; white; insurance type--Medicare Advantage (MA) plan with a visit-based reimbursement, MA plan with an episode-based reimbursement, dual eligible with Medicaid enrollment (reference group: Medicare FFS); dummy for having no assistance available; dummy for living alone.
	In column (4), comorbidity controls include dummies for 17 Charlson comorbidity index factors; dummies overall status--very bad (patient has serious progressive conditions that could lead to death within a year), bad (patient is likely to remain in fragile health); and temporarily bad (temporary facing high health risks); dummies for high-risk factors--alcohol dependency, drug dependency, smoking, obesity--and dummies for conditions prior to hospital stay within past 14 days--disruptive or socially inappropriate behavior, impaired decision making, indwelling or suprapublic catheter, intractable pain, serious memory loss, urinary incontinence.
	 *significant at 10\%; **significant at 5\%; ***significant at 1\%.
	\end{tablenotes}
\end{threeparttable}
\end{table}



\clearpage
\subsection{Correlation of Selected Measures of Patients' Severity and Rehospitalization}
\label{appendix:corr_severity}

Table~\ref{tab:ols_corr_severity} presents coefficient estimates on three selected measures of patients' severity---indicators for each category of Charlson comorbidity index, overall status likely to remain fragile, and taking 5 or more medications---obtained from estimating the model in Column (4) of Table~\ref{tab:ols_handoffonly}.
We find that reported severity measures are statistically significant and strong predictors of the likelihood of readmission, even stronger than handoff.

\begin{table}[H]
\footnotesize
\setlength\tabcolsep{0pt}
\centering
\caption{Key Measures of Patients' Severity as Predictors of the Likelihood of Readmission}
\label{tab:ols_corr_severity}
\begin{threeparttable}
{
\begin{tabular}{lc} \toprule
 & Dep Var: Indicator for being rehospitalized \\
\midrule
Handoff & 0.0017*** \\
 & (0.0002) \\
\\
\multicolumn{2}{l}{A. 17 Components of the Charlson Comorbidity Index}\\
Acute myocardial infarction (AMI) & 0.0001 \\
 & (0.0007) \\
Congestive heart failure (CHF) & 0.0028*** \\
 & (0.0003) \\
Peripheral vascular disease (PVD) & 0.0026*** \\
 & (0.0009) \\
Cerebrovascular disease (CEVD) & 0.0008** \\
 & (0.0003) \\
Dementia & 0.0018* \\
 & (0.0011) \\
Chronic pulmonary disease (COPD) & 0.0023*** \\
 & (0.0004) \\
Rheumatic disease & 0.0008 \\
 & (0.0025) \\
Peptic ulcer disease & 0.0052** \\
 & (0.0020) \\
Mild liver disease & 0.0016 \\
 & (0.0016) \\
Diabetes & 0.0014** \\
 & (0.0007) \\
Diabetes + Complications & 0.0057*** \\
 & (0.0013) \\
Hemiplegia or paraplegia (HP/PAPL) & 0.0002 \\
 & (0.0017) \\
Renal disease & 0.0032*** \\
 & (0.0007) \\
Cancer & 0.0041*** \\
 & (0.0005) \\
Moderate/severe liver disease & 0.0067*** \\
 & (0.0024) \\
Metastatic cancer & 0.0045*** \\
 & (0.0015) \\
AIDS/HIV & 0.0107 \\
 & (0.0115) \\
\\
\multicolumn{2}{l}{B. Overall Status}\\
Likely to remain in fragile health & 0.0034*** \\
 & (0.0004) \\
\\
\multicolumn{2}{l}{C. Risk for Hospitalization}\\
Take 5 or more medications & 0.0005* \\
 & (0.0003) \\
R-squared & 0.0081 \\
Observations & 1,031,904 \\
\bottomrule
\end{tabular}
}
	\begin{tablenotes}[para,flushleft]
	\scriptsize
	\item \emph{Notes.}
	An observation is a patient episode-day. Standard errors are clustered at the home health office level, allowing for arbitrary correlation among episode-days in the same office in parentheses.
	*significant at 10\%; **significant at 5\%; ***significant at 1\%.
	\end{tablenotes}
\end{threeparttable}
\end{table}


\clearpage
\subsection{Robustness Check Using an Alternative Definition of Breaks in Nurses' Availability} \label{appendix:lws_dist_alt}

In this appendix, we run a robustness check on our main IV results using an alternative definition of medium absence.
We define medium absence as not providing visits in any office for 6 to 20 consecutive days, and accordingly, short absence for 1 to 5 consecutive days and long absence for 21 or more consecutive days.

Table~\ref{tab:lws_dist_alt} presents the distribution of the number of patient episode-day observations as well as the likelihood of having a provider handoff, a nurse visit, and a hospital readmission for each newly defined availability category.
The distributions of observations and probabilities of handoffs, nurse visits, and readmissions change little when we use a wider window of time for medium absence.
A noticeable difference is an increase in the probabilities of handoffs and readmissions when a provider is having a long absence under the new definition.
However, qualitatively, the relative orders of these numbers remain the same across the categories.
Therefore, we obtain very similar first-stage and second-stage estimates in Table~\ref{tab:iv_robust} to those in Table~\ref{tab:iv}.
In Panels A and B for the first-stage results, each of providers' unavailability status seems to be a stronger predictor of patients' handoffs and a slightly weaker predictor of patients' receiving nurse visits.
However, the F-statistic values are still significantly large.
In Panel C, in Column (4), we find a tiny decrease in the magnitude of the effect of experiencing a handoff at 0.37 percentage points or 53\% on the likelihood of rehospitalization.


\begin{table}[H]
\centering
\footnotesize
\caption{Distribution of Patient-Day Observations and the Likelihood of Nurse Handoff, Nurse Visit, and Readmission by the Availability of Nurse Who Visited a Patient in the Last Visit}
\label{tab:lws_dist_alt}
\begin{threeparttable}
\begin{tabular}{lccccc}
\toprule
& N Obs                    & \% Obs & \% Handoff & \% Have a nurse visit & \% Readmission   \\
\midrule
Active                   & 670,621 & 64.99      & 20.63            & 27.37          & 0.77 \\
Short absence (1-5 days) & 280,872 & 27.22      & 29.95            & 5.94           & 0.55 \\
Medium absence (6-20 days)     & 47,559  & 4.61       & 63.72            & 12.22          & 0.70 \\
Long absence (21+ days)  & 7,733  & 0.75       & 77.42            & 11.95          & 0.72 \\
Assigned to other office     & 13,415  & 1.30       & 49.30            & 11.20          & 0.83 \\
Attrition           & 11,704  & 1.13       & 65.02            & 9.28           & 0.67 \\
\midrule
Total                    & 1,031,904 & 100.00      &           &         \\
\bottomrule
\end{tabular}
	\begin{tablenotes}[para,flushleft]
	\footnotesize
	\item \emph{Notes.} In the entire sample of patient-day observations, the percentage of handoff is 26.46\%; the percentage of having a nurse visit is 20.31\%; the percentage of readmission is 0.71\%.
\end{tablenotes}
\end{threeparttable}
\end{table}


\begin{table}[H]
\footnotesize
\setlength\tabcolsep{0pt}
\centering
\caption{The Effect of Handoffs on the Likelihood of Rehospitalization When Using an Alternative Definition of Breaks in Nurses' Availability}
\label{tab:iv_robust}
\begin{threeparttable}
{
\def\sym#1{\ifmmode^{#1}\else\(^{#1}\)\fi}
\begin{tabular*}{\textwidth}{l@{\extracolsep{\fill}}*{4}{c}}
\toprule
                    &\multicolumn{1}{c}{(1)}&\multicolumn{1}{c}{(2)}&\multicolumn{1}{c}{(3)}&\multicolumn{1}{c}{(4)}\\
\midrule
\multicolumn{5}{l}{A. First stage - Handoff} \\
Short absence       &      0.1190***&      0.1190***&      0.1190***&      0.1190***\\
                    &    (0.0061)   &    (0.0061)   &    (0.0061)   &    (0.0060)   \\
Medium absence      &      0.4232***&      0.4232***&      0.4231***&      0.4227***\\
                    &    (0.0101)   &    (0.0101)   &    (0.0101)   &    (0.0101)   \\
Long absence        &      0.5573***&      0.5574***&      0.5568***&      0.5568***\\
                    &    (0.0190)   &    (0.0190)   &    (0.0191)   &    (0.0190)   \\
Other office        &      0.2995***&      0.2996***&      0.2993***&      0.2994***\\
                    &    (0.0206)   &    (0.0206)   &    (0.0206)   &    (0.0206)   \\
Attrition           &      0.4571***&      0.4574***&      0.4578***&      0.4575***\\
                    &    (0.0156)   &    (0.0156)   &    (0.0156)   &    (0.0156)   \\
R-squared           &       0.178   &       0.178   &       0.178   &       0.179   \\
F-statistic         &     439.211   &     439.030   &     440.619   &     442.788   \\
\\
\multicolumn{5}{l}{B. First stage - Have a nurse visit} \\
Short absence       &     -0.1111***&     -0.1112***&     -0.1112***&     -0.1112***\\
                    &    (0.0048)   &    (0.0048)   &    (0.0048)   &    (0.0048)   \\
Medium absence      &     -0.0956***&     -0.0957***&     -0.0957***&     -0.0957***\\
                    &    (0.0032)   &    (0.0032)   &    (0.0033)   &    (0.0033)   \\
Long absence        &     -0.0945***&     -0.0945***&     -0.0944***&     -0.0943***\\
                    &    (0.0046)   &    (0.0046)   &    (0.0046)   &    (0.0046)   \\
Other office        &     -0.1321***&     -0.1322***&     -0.1320***&     -0.1320***\\
                    &    (0.0071)   &    (0.0072)   &    (0.0072)   &    (0.0072)   \\
Attrition           &     -0.0785***&     -0.0783***&     -0.0783***&     -0.0779***\\
                    &    (0.0041)   &    (0.0041)   &    (0.0042)   &    (0.0042)   \\
R-squared           &       0.244   &       0.244   &       0.245   &       0.245   \\
F-statistic         &     234.768   &     232.472   &     230.006   &     226.544   \\
\\
\multicolumn{5}{l}{C. Second stage - Rehospitalization} \\
Handoff             &      0.0035***&      0.0036***&      0.0037***&      0.0037***\\
                    &    (0.0009)   &    (0.0009)   &    (0.0009)   &    (0.0009)   \\
Have a nurse visit  &      0.0036   &      0.0038   &      0.0039   &      0.0040   \\
                    &    (0.0025)   &    (0.0025)   &    (0.0025)   &    (0.0024)   \\
R-squared           &       0.004   &       0.005   &       0.005   &       0.006   \\
J-statistic p-value &       0.626   &       0.571   &       0.547   &       0.617   \\
\midrule
Observations        &     1,031,904   &     1,031,904   &     1,031,904   &     1,031,904   \\
Hospitalization risk controls & . & Yes & Yes & Yes \\
Demographic controls & . & . & Yes & Yes \\
 Comorbidity controls & . & . & . & Yes \\
\bottomrule
\end{tabular*}
}
	\begin{tablenotes}[para,flushleft]
	\scriptsize
		\item \emph{Notes.} This table presents IV estimates of the effect of handoffs on the likelihood of rehospitalization obtained with a two-step efficient generalized method of moments (GMM) estimator.
	We instrument the indicator variables for skilled nurse handoffs and for having a nurse visit with the following 5 instruments:
(1) Active--visiting patients in the patient's home office;
(2) Short absence--not providing visits in any office for 1 to 4 consecutive days;
(3) Medium absence--not providing visits in any office for 6 to 20 consecutive days;
(4) Long absence--not providing visits in any office for 21 or more consecutive days;
(5) Assigned to other office--providing visits exclusively in a different office; and
(6) Attrition--day post labor termination for nurse (due to either quit or layoff) according to HR records.
	An observation is a patient episode-day.
	Standard errors are clustered at the home health office level, allowing for arbitrary correlation among episode-days in the same office in parentheses.
	In all columns, regressions include mean interval of days between two consecutive visits during the episode, number of days since last nurse visit for the patient, number of days since last visit by any provider for the patient; number of ongoing episodes in the office-day, number of nurses working in the office-day; office fixed effects, day of week fixed effects, month-year of the day fixed effects, home health day fixed effects; and hospitalization risk controls, demographic controls, and comorbidity controls.
	*significant at 10\%; **significant at 5\%; ***significant at 1\%.

	\end{tablenotes}
\end{threeparttable}
\end{table}


\subsection{Robustness Check with Some Dynamic Patient-Severity Covariates Excluded} \label{appendix:iv_noendog}

\clearpage
\begin{table}[H]
\footnotesize
\setlength\tabcolsep{0pt}
\centering
\caption{The Effect of Handoffs on the Likelihood of Rehospitalization with Some Dynamic Patient-Severity Covariates Excluded}
\label{tab:iv_noendog}
\begin{threeparttable}
{
\def\sym#1{\ifmmode^{#1}\else\(^{#1}\)\fi}
\begin{tabular*}{\textwidth}{l@{\extracolsep{\fill}}*{4}{c}}
\toprule
                    &\multicolumn{1}{c}{(1)}&\multicolumn{1}{c}{(2)}&\multicolumn{1}{c}{(3)}&\multicolumn{1}{c}{(4)}\\
\midrule
\multicolumn{5}{l}{A. First stage - Handoff} \\
Short absence       &      0.1328***&      0.1328***&      0.1328***&      0.1327***\\
                    &    (0.0064)   &    (0.0064)   &    (0.0064)   &    (0.0063)   \\
Medium absence      &      0.4128***&      0.4128***&      0.4126***&      0.4123***\\
                    &    (0.0101)   &    (0.0101)   &    (0.0101)   &    (0.0100)   \\
Long absence        &      0.5338***&      0.5340***&      0.5334***&      0.5330***\\
                    &    (0.0119)   &    (0.0119)   &    (0.0120)   &    (0.0120)   \\
Other office        &      0.3012***&      0.3012***&      0.3010***&      0.3011***\\
                    &    (0.0207)   &    (0.0208)   &    (0.0207)   &    (0.0207)   \\
Attrition           &      0.4611***&      0.4614***&      0.4619***&      0.4615***\\
                    &    (0.0154)   &    (0.0154)   &    (0.0154)   &    (0.0154)   \\
R-squared           &       0.173   &       0.173   &       0.173   &       0.174   \\
F-statistic         &     556.207   &     554.149   &     553.745   &     555.617   \\
\multicolumn{5}{l}{B. First stage - Have a nurse visit} \\
Short absence       &     -0.1185***&     -0.1184***&     -0.1184***&     -0.1185***\\
                    &    (0.0052)   &    (0.0052)   &    (0.0052)   &    (0.0052)   \\
Medium absence      &     -0.1104***&     -0.1104***&     -0.1104***&     -0.1103***\\
                    &    (0.0035)   &    (0.0035)   &    (0.0035)   &    (0.0035)   \\
Long absence        &     -0.1147***&     -0.1145***&     -0.1143***&     -0.1141***\\
                    &    (0.0058)   &    (0.0057)   &    (0.0057)   &    (0.0058)   \\
Other office        &     -0.1514***&     -0.1511***&     -0.1514***&     -0.1515***\\
                    &    (0.0073)   &    (0.0073)   &    (0.0073)   &    (0.0074)   \\
Attrition           &     -0.1149***&     -0.1147***&     -0.1149***&     -0.1152***\\
                    &    (0.0048)   &    (0.0048)   &    (0.0047)   &    (0.0046)   \\
R-squared           &       0.174   &       0.175   &       0.175   &       0.175   \\
F-statistic         &     258.931   &     257.414   &     256.710   &     262.303   \\
\multicolumn{5}{l}{C. Second stage - Rehospitalization} \\
Handoff             &      0.0023** &      0.0025** &      0.0026** &      0.0027** \\
                    &    (0.0011)   &    (0.0011)   &    (0.0011)   &    (0.0011)   \\
Have a nurse visit  &      0.0026   &      0.0026   &      0.0027   &      0.0029   \\
                    &    (0.0026)   &    (0.0026)   &    (0.0026)   &    (0.0026)   \\
R-squared           &       0.001   &       0.001   &       0.001   &       0.002   \\
J-statistic p-value &       0.813   &       0.747   &       0.733   &       0.751   \\
\midrule
Observations        &     1,031,904   &     1,031,904   &     1,031,904   &     1,031,904   \\
Hospitalization risk controls & . & Yes & Yes & Yes \\
Demographic controls & . & . & Yes & Yes \\
 Comorbidity controls & . & . & . & Yes \\
\bottomrule
\end{tabular*}
}
	\begin{tablenotes}[para,flushleft]
	\scriptsize
		\item \emph{Notes.} This table presents IV estimates of the effect of handoffs on the likelihood of rehospitalization obtained with a two-step efficient generalized method of moments (GMM) estimator.
	We instrument the indicator variables for skilled nurse handoffs and for having a nurse visit with the following 5 instruments:
(1) Active--visiting patients in the patient's home office;
(2) Short absence--not providing visits in any office for 1 to 4 consecutive days;
(3) Medium absence--not providing visits in any office for 6 to 20 consecutive days;
(4) Long absence--not providing visits in any office for 21 or more consecutive days;
(5) Assigned to other office--providing visits exclusively in a different office; and
(6) Attrition--day post labor termination for nurse (due to either quit or layoff) according to HR records.
	An observation is a patient episode-day.
	Standard errors are clustered at the home health office level, allowing for arbitrary correlation among episode-days in the same office in parentheses.
	In all columns, regressions include number of days since last nurse visit for the patient; number of ongoing episodes in the office-day, number of nurses working in the office-day; office fixed effects, day of week fixed effects, month-year of the day fixed effects, home health day fixed effects; and hospitalization risk controls, demographic controls, and comorbidity controls.
	*significant at 10\%; **significant at 5\%; ***significant at 1\%.

	\end{tablenotes}
\end{threeparttable}
\end{table}


\clearpage
\subsection{Robustness Check Using a Subset of Patients Who Started Home Health Episode within 3 Days from Hospital Discharge} \label{appendix:dist_days_fromhospdc}

As a robustness check, we re-estimate our main specification (Table~\ref{tab:iv}) using the subset of patients who started home health episode within 3 days of hospital discharge.\footnote{Unfortunately, the date of hospital discharge is missing from 3\% of episodes (42,432 compared with 43,740 episodes).} The results reported below in Table~\ref{tab:iv_days_fromhospdc} are effectively the same as those in the original Table~\ref{tab:iv} (with 904,724 observation as opposed to 1,031,904 observations). This result is not surprising as the distribution of days from hospital discharge to first home health visit is highly right-skewed (see Table~\ref{tab:dist_days_fromhospdc} below). Note that two thirds of home health episodes started within 1 day of hospital discharge and approximately 91\% of home health episodes started within 3 days of hospital discharge. Among the remaining 9\% of episodes, over 7\% of episodes began within a week of hospital discharge.



\begin{table}[H]
\footnotesize
\setlength\tabcolsep{0pt}
\centering
\caption{The Effect of Handoffs on the Likelihood of Rehospitalization among Patients Who Started Home Health Episode within 3 Days from Hospital Discharge}
\label{tab:iv_days_fromhospdc}
\begin{threeparttable}
{
\def\sym#1{\ifmmode^{#1}\else\(^{#1}\)\fi}
\begin{tabular*}{\textwidth}{l@{\extracolsep{\fill}}*{4}{c}}
\toprule
                    &\multicolumn{1}{c}{(1)}&\multicolumn{1}{c}{(2)}&\multicolumn{1}{c}{(3)}&\multicolumn{1}{c}{(4)}\\
\midrule
\multicolumn{5}{l}{A. First stage - Handoff} \\
Short absence       &      0.1361***&      0.1361***&      0.1361***&      0.1361***\\
                    &    (0.0063)   &    (0.0063)   &    (0.0063)   &    (0.0063)   \\
Medium absence      &      0.4156***&      0.4156***&      0.4156***&      0.4151***\\
                    &    (0.0102)   &    (0.0102)   &    (0.0101)   &    (0.0101)   \\
Long absence        &      0.5312***&      0.5313***&      0.5309***&      0.5306***\\
                    &    (0.0119)   &    (0.0119)   &    (0.0119)   &    (0.0119)   \\
Other office        &      0.3027***&      0.3028***&      0.3027***&      0.3027***\\
                    &    (0.0212)   &    (0.0212)   &    (0.0211)   &    (0.0211)   \\
Attrition           &      0.4611***&      0.4612***&      0.4617***&      0.4615***\\
                    &    (0.0160)   &    (0.0161)   &    (0.0160)   &    (0.0160)   \\
R-squared           &       0.176   &       0.176   &       0.177   &       0.177   \\
F-statistic         &     580.552   &     579.037   &     576.186   &     580.073   \\
\\
\multicolumn{5}{l}{B. First stage - Have a nurse visit} \\
Short absence       &     -0.1112***&     -0.1112***&     -0.1112***&     -0.1112***\\
                    &    (0.0047)   &    (0.0047)   &    (0.0047)   &    (0.0047)   \\
Medium absence      &     -0.0971***&     -0.0971***&     -0.0972***&     -0.0971***\\
                    &    (0.0036)   &    (0.0036)   &    (0.0036)   &    (0.0036)   \\
Long absence        &     -0.0958***&     -0.0958***&     -0.0958***&     -0.0958***\\
                    &    (0.0047)   &    (0.0048)   &    (0.0048)   &    (0.0048)   \\
Other office        &     -0.1327***&     -0.1327***&     -0.1325***&     -0.1325***\\
                    &    (0.0067)   &    (0.0067)   &    (0.0068)   &    (0.0067)   \\
Attrition           &     -0.0778***&     -0.0778***&     -0.0778***&     -0.0773***\\
                    &    (0.0045)   &    (0.0045)   &    (0.0046)   &    (0.0045)   \\
R-squared           &       0.247   &       0.247   &       0.247   &       0.247   \\
F-statistic         &     189.250   &     188.446   &     186.457   &     185.237   \\
\\
\multicolumn{5}{l}{C. Second stage - Rehospitalization} \\
Handoff             &      0.0040***&      0.0041***&      0.0042***&      0.0042***\\
                    &    (0.0012)   &    (0.0012)   &    (0.0012)   &    (0.0012)   \\
Have a nurse visit  &      0.0043   &      0.0044   &      0.0046*  &      0.0045*  \\
                    &    (0.0027)   &    (0.0027)   &    (0.0027)   &    (0.0027)   \\
R-squared           &       0.004   &       0.005   &       0.005   &       0.006   \\
J-statistic p-value &       0.631   &       0.543   &       0.531   &       0.596   \\
\\
\midrule
Observations        &      904,724   &      904,724   &      904,724   &      904,724   \\
Hospitalization risk controls & . & Yes & Yes & Yes \\
Demographic controls & . & . & Yes & Yes \\
 Comorbidity controls & . & . & . & Yes \\
\bottomrule
\end{tabular*}
}
	\begin{tablenotes}[para,flushleft]
	\scriptsize
	\item \emph{Notes.}  This table presents IV estimates of the effect of handoffs on the likelihood of rehospitalization obtained with a two-step efficient generalized method of moments (GMM) estimator.
	We instrument the indicator variables for nurse handoffs and for nurse visits with the following 5 instruments:
(1) Active--visiting patients in the patient's home office;
(2) Short absence--not providing visits in any office for 1 to 6 consecutive days;
(3) Medium absence--not providing visits in any office for 7 to 14 consecutive days;
(4) Long absence--not providing visits in any office for 15 or more consecutive days;
(5) Assigned to other office--providing visits exclusively in a different office; and
(6) Attrition--day post labor termination for nurse according to HR records.
	An observation is a patient episode-day.
	Robust standard errors allowing for arbitrary correlation within the same office in parentheses.
		In all specifications, we control for the cumulative number of nurse visits provided, mean interval of days between two consecutive visits, number of days since last nurse visit, number of days since last visit by any provider; number of ongoing episodes in the office-day, number of nurses working in the office-day; and office fixed effects, day of week fixed effects, month-year of the day fixed effects, and home health day fixed effects.
	Columns (2)-(4) additionally control for hospitalization risk controls, demographic controls, and comorbidity controls, respectively.
	*significant at 10\%; **significant at 5\%; ***significant at 1\%.

	\end{tablenotes}
\end{threeparttable}
\end{table}



\begin{table}[H]
\footnotesize
\centering
\caption{Distribution of the Number of Days from Hospital Discharge}
\label{tab:dist_days_fromhospdc}
\begin{tabular}{S[table-format=2]S[table-format=5]S[table-format=2.2]S[table-format=3.2]}
\toprule
{Days}	&	{Frequency}	&	{Percent}	&	{Cumulative}	\\
\midrule
0	&	1374	&	3.24	&	3.24	\\
1	&	26538	&	62.54	&	65.78	\\
2	&	7788	&	18.35	&	84.13	\\
3	&	2862	&	6.74	&	90.88	\\
4	&	1379	&	3.25	&	94.13	\\
5	&	744	&	1.75	&	95.88	\\
6	&	441	&	1.04	&	96.92	\\
7	&	368	&	0.87	&	97.79	\\
8	&	261	&	0.62	&	98.4	\\
9	&	179	&	0.42	&	98.83	\\
10	&	151	&	0.36	&	99.18	\\
11	&	111	&	0.26	&	99.44	\\
12	&	89	&	0.21	&	99.65	\\
13{+}	&	147	&	0.35	&	100.00	\\
\midrule
{Total}	&	42432	& &				\\
\bottomrule
\end{tabular}
\end{table}


\end{singlespace}


\end{document}
